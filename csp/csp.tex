% Created 2021-09-06 Mon 15:52
% Intended LaTeX compiler: pdflatex
\documentclass{scrartcl}
\usepackage[utf8]{inputenc}
\usepackage[T1]{fontenc}
\usepackage{fontspec}
\usepackage{graphicx}
\usepackage{grffile}
\usepackage{longtable}
\usepackage{wrapfig}
\usepackage{rotating}
\usepackage[normalem]{ulem}
\usepackage{amsmath}
\usepackage{textcomp}
\usepackage{amssymb}
\usepackage{capt-of}
\usepackage[dvipsnames]{xcolor}
\usepackage[colorlinks=true, linkcolor=Blue, citecolor=BrickRed, urlcolor=PineGreen]{hyperref}
\usepackage{indentfirst}
\setmainfont[Ligatures=TeX]{Alegreya}
\setmonofont[Ligatures=TeX]{Liga SFMono Nerd Font}
% features: (acronym par-sep)
\newcommand{\acr}[1]{\protect\textls*[110]{\scshape #1}}
\newcommand{\acrs}{\protect\scalebox{.91}[.84]{\hspace{0.15ex}s}}
\setlength{\parskip}{\baselineskip}
\setlength{\parindent}{0pt}

% end features

%% make document follow Emacs theme

\definecolor{obg}{HTML}{fafafa}
\definecolor{ofg}{HTML}{383a42}

\pagecolor{obg}
\color{ofg}

% list labels

\definecolor{itemlabel}{HTML}{4078f2}

\renewcommand{\labelitemi}{\textcolor{itemlabel}{\textbullet}}
\renewcommand{\labelitemii}{\textcolor{itemlabel}{\normalfont\bfseries \textendash}}
\renewcommand{\labelitemiii}{\textcolor{itemlabel}{\textasteriskcentered}}
\renewcommand{\labelitemiv}{\textcolor{itemlabel}{\textperiodcentered}}

\renewcommand{\labelenumi}{\textcolor{itemlabel}{\theenumi.}}
\renewcommand{\labelenumii}{\textcolor{itemlabel}{(\theenumii)}}
\renewcommand{\labelenumiii}{\textcolor{itemlabel}{\theenumiii.}}
\renewcommand{\labelenumiv}{\textcolor{itemlabel}{\theenumiv.}}

% structural elements

\definecolor{documentTitle}{HTML}{a626a4}
\definecolor{documentInfo}{HTML}{a626a4}
\definecolor{level1}{HTML}{e45649}
\definecolor{level2}{HTML}{da8548}
\definecolor{level3}{HTML}{b751b6}
\definecolor{level4}{HTML}{6f99f5}
\definecolor{level5}{HTML}{bc5cba}
\definecolor{level6}{HTML}{9fbbf8}
\definecolor{level7}{HTML}{d292d1}
\definecolor{level8}{HTML}{d8e4fc}

\addtokomafont{title}{\color{documentTitle}}
\addtokomafont{author}{\color{documentInfo}}
\addtokomafont{date}{\color{documentInfo}}
\addtokomafont{section}{\color{level1}}
\newkomafont{sectionprefix}{\color{level1}}
\addtokomafont{subsection}{\color{level2}}
\newkomafont{subsectionprefix}{\color{level2}}
\addtokomafont{subsubsection}{\color{level3}}
\newkomafont{subsubsectionprefix}{\color{level3}}
\addtokomafont{paragraph}{\color{level4}}
\newkomafont{paragraphprefix}{\color{level4}}
\addtokomafont{subparagraph}{\color{level5}}
\newkomafont{subparagraphprefix}{\color{level5}}

% textual elements

\definecolor{link}{HTML}{4078f2}
\definecolor{cite}{HTML}{800080}
\definecolor{itemlabel}{HTML}{4078f2}
\definecolor{code}{HTML}{da8548}
\definecolor{verbatim}{HTML}{50a14f}

\renewcommand{\labelitemi}{\textcolor{itemlabel}{\textbullet}}
\renewcommand{\labelitemii}{\textcolor{itemlabel}{\normalfont\bfseries \textendash}}
\renewcommand{\labelitemiii}{\textcolor{itemlabel}{\textasteriskcentered}}
\renewcommand{\labelitemiv}{\textcolor{itemlabel}{\textperiodcentered}}

\renewcommand{\labelenumi}{\textcolor{itemlabel}{\theenumi.}}
\renewcommand{\labelenumii}{\textcolor{itemlabel}{(\theenumii)}}
\renewcommand{\labelenumiii}{\textcolor{itemlabel}{\theenumiii.}}
\renewcommand{\labelenumiv}{\textcolor{itemlabel}{\theenumiv.}}

\DeclareTextFontCommand{\texttt}{\color{code}\ttfamily}
\makeatletter
\def\verbatim@font{\color{verbatim}\normalfont\ttfamily}
\makeatother

% code blocks

\definecolor{codebackground}{HTML}{fafafa}
\colorlet{EFD}{ofg}
\definecolor{codeborder}{HTML}{f0f0f0}

%% end customisations

\author{Shaurya Singh}
\date{\today}
\title{Ap CSP Summer Assignment}
\colorlet{greenyblue}{blue!70!green}
\colorlet{blueygreen}{blue!40!green}
\providecolor{link}{named}{greenyblue}
\providecolor{cite}{named}{blueygreen}
\hypersetup{
  pdfauthor={Shaurya Singh},
  pdftitle={Ap CSP Summer Assignment},
  pdfkeywords={},
  pdfsubject={},
  pdfcreator={Emacs 28.0.50 (Org mode 9.5)},
  pdflang={English},
  breaklinks=true,
  colorlinks=true,
  linkcolor=,
  urlcolor=link,
  citecolor=cite
}
\urlstyle{same}
\begin{document}

\maketitle
\setcounter{tocdepth}{2}
\tableofcontents


\section{Digital Explosion}
\label{sec:orgfdad440}
\subsection{Vocabulary}
\label{sec:orgfbd6f64}
\begin{enumerate}
\item Word: bit

Definition: (n) \textbf{bit} (a unit of measurement of information (from binary + digit); the amount of information in a system having two equiprobable states) ``there are 8 bits in a byte''

Ref: WordNet Search - 3.1. \url{http://wordnetweb.princeton.edu/perl/webwn?s=bit\&sub=Search+WordNet\&o2=\&o0=1\&o8=1\&o1=1\&o7=\&o5=\&o9=\&o6=\&o3=\&o4=\&h}=. Accessed 18 July 2021.

\item Word: koan

Definition: (n) \textbf{koan} (a paradoxical anecdote or a riddle that has no solution; used in Zen Buddhism to show the inadequacy of logical reasoning)

Ref: WordNet Search - 3.1. \url{http://wordnetweb.princeton.edu/perl/webwn?s=koan\&sub=Search+WordNet\&o2=\&o0=1\&o8=1\&o1=1\&o7=\&o5=\&o9=\&o6=\&o3=\&o4=\&h}=. Accessed 18 July 2021.

\item Word: ping

Definition: (v) \textbf{ping} (send a message from one computer to another to check whether it is reachable and active) ``ping your machine in the office''

Ref: WordNet Search - 3.1. \url{http://wordnetweb.princeton.edu/perl/webwn?s=Ping\&sub=Search+WordNet\&o2=\&o0=1\&o8=1\&o1=1\&o7=\&o5=\&o9=\&o6=\&o3=\&o4=\&h}=. Accessed 23 July 2021.

\item Word: benign

Definition: (adj) \textbf{benign} (not dangerous to health; not recurrent or progressive (especially of a tumor))

Ref: WordNet Search - 3.1. \url{http://wordnetweb.princeton.edu/perl/webwn?s=benign\&sub=Search+WordNet\&o2=\&o0=1\&o8=1\&o1=1\&o7=\&o5=\&o9=\&o6=\&o3=\&o4=\&h=0000000}. Accessed 23 July 2021.

\item Word: utopian

Definition: (n) \textbf{utopian} (an idealistic (but usually impractical) social reformer) ``a Utopian believes in the ultimate perfectibility of man''

Ref: WordNet Search - 3.1. \url{http://wordnetweb.princeton.edu/perl/webwn?s=utopian\&sub=Search+WordNet\&o2=\&o0=1\&o8=1\&o1=1\&o7=\&o5=\&o9=\&o6=\&o3=\&o4=\&h=0000000}. Accessed 23 July 2021.

\item Word: err

Definition: (v) \textbf{err}, mistake, slip (to make a mistake or be incorrect)

Ref: WordNet Search - 3.1. \url{http://wordnetweb.princeton.edu/perl/webwn?s=err\&sub=Search+WordNet\&o2=\&o0=1\&o8=1\&o1=1\&o7=\&o5=\&o9=\&o6=\&o3=\&o4=\&h=0000000}. Accessed 23 July 2021.

\item Word: paradoxically

Definition: (adv) \textbf{paradoxically} (in a paradoxical manner) ``paradoxically,  ice ages seem to occur when the sun gets hotter''

(adj) \textbf{paradoxical}, self-contradictory (seemingly contradictory but nonetheless possibly true) ``it is paradoxical that standing is more tiring than walking''

Ref: WordNet Search - 3.1. \url{http://wordnetweb.princeton.edu/perl/webwn?s=paradoxically\&sub=Search+WordNet\&o2=\&o0=1\&o8=1\&o1=1\&o7=\&o5=\&o9=\&o6=\&o3=\&o4=\&h=0000000}. Accessed 23 July 2021.

\item Word: expunge

Definition: (v)  \textbf{expunge} (remove by erasing or crossing out or as if by drawing a line)

Ref: WordNet Search - 3.1.
\url{http://wordnetweb.princeton.edu/perl/webwn?s=expunge\&sub=Search+WordNet\&o2=\&o0=1\&o8=1\&o1=1\&o7=\&o5=\&o9=\&o6=\&o3=\&o4=\&h=0000000}. Accessed 23 July 2021.

\item Word: database

Definition: (n) \textbf{database} (an organized body of related information)

Ref: WordNet Search - 3.1.
\url{http://wordnetweb.princeton.edu/perl/webwn?s=database\&sub=Search+WordNet\&o2=\&o0=1\&o8=1\&o1=1\&o7=\&o5=\&o9=\&o6=\&o3=\&o4=\&h=0000000}. Accessed 23 July 2021.

\item Word: blacklist

Definition: (v) \textbf{blacklist} (put on a blacklist so as to banish or cause to be boycotted) ``many books were blacklisted by the Nazis''

Ref: WordNet Search - 3.1.
\url{http://wordnetweb.princeton.edu/perl/webwn?s=blacklist\&sub=Search+WordNet\&o2=\&o0=1\&o8=1\&o1=1\&o7=\&o5=\&o9=\&o6=\&o3=\&o4=\&h=0000000}. Accessed 23 July 2021.
\end{enumerate}

\subsection{Ideas:}
\label{sec:orga0f1f5d}
\begin{enumerate}
\item Companies keep records of cellphone locations (Page 1)
\item ``Its all just bits'' (Page 5)
\item ``In fact, processors have hardly grown faster at all'' (Page 8)
\item ``By 2011, we may be producing more bits than we can store'' (Page 10)
\end{enumerate}

\subsection{Journal Entry}
\label{sec:orgddd5a65}
\begin{enumerate}
\item I support the idea of companies keeping track of cellphone locations, as long
as that information is kept private and in the hands of only the government.
Cellular locations are incredibly useful for finding missing people, and
quickly reacting to emergencies, such as in Tanya's case.

However, the key term is \emph{as long as that information is kept private and in
the hands of the government}. Private companies shouldn't be able to get the
locations of users. These locations are often sold to advertising firms.
Companies like Cuebiq make money by collecting location data from smartphone
users who agree to share their locations for weather or maps, then analyse
and sell that data to advertisers and marketers. Location data shouldn't be  used to target people, and should be used as a last resort

\item While technically it is all just bits, personally I believe we shouldn't
think of it as such. Thinking of computers in terms of bits is like thinking
of writing in terms of atoms on a piece of paper. While all handwritten work
is technically just graphite on paper, we interpret it as much more than
that. We think of writing in terms of words, phrases, paragraphs, and should
think of the computer experience in terms of experiences. Similarly,
copyright law is based on text, and so laws pertaining to computers should be
based on the end user's experience, not what delivers that experience

\item Considering the next sentence is talking about ``multiple processors on the
same chip'' I assume this statement is talking about the processor cores
themselves. However, data shows that the fastest processors today are about
80 times faster in terms of single core performance with the same efficiency.
Still, it's true that processor cores have increased over the years (from
single core chips to server chips with over 128 cores in the span of just 20
years).

In my opinion we should focus more on processor efficiency and less on raw
processor speed. Processor's these days, even budget ones, are more than fast
enough for the majority of use cases. The market for large, high end HEDT
processors is extremely small compared to the market share of their lower end
counterparts. The future is about switching to arm-based processors in
laptops, which should give much more performance at the same wattage.

\item I found this statement the most interesting in this chapter. We've made leaps
and bounds in storage technology since then, nowadays you can buy tens of
terabytes of storage for quite cheap, compared to the expensive ``high end''
80gb drives you could get in 2008, back when this textbook
released.

Similar my opinion on processor speed, instead of focusing on storage size
and how much data we can store, we should talk about how we store data.
Companies are moving to cloud-based centralized storage, and as of 2020 50\%
of all corporate data is stored in the cloud, up from 30\% just 5 years
earlier. As more and more people start using cloud services (e.g. OneDrive,
iCloud, Google Drive), we should focus on make data transfer to and from
those drives quicker and more secure.
\end{enumerate}

\section{Naked in the Sunlight}
\label{sec:org72a140a}
\subsection{Vocabulary}
\label{sec:orgad53fd9}
\begin{enumerate}
\item Word: pervasive

Definition: per·va·sive | pərˈvāsiv | adjective (especially of an unwelcome influence or physical effect) spreading widely throughout an area or a group of people: ageism is pervasive and entrenched in our society.

Ref: WordNet Search - 3.1. \url{http://wordnetweb.princeton.edu/perl/webwn?s=pervasive\&sub=Search+WordNet\&o2=\&o0=1\&o8=1\&o1=1\&o7=\&o5=\&o9=\&o6=\&o3=\&o4=\&h}=. Accessed 18 July 2021.

\item Word: cleric

Definition: cler·ic | ˈklerik | noun a priest or religious leader, especially a Christian or Muslim one.

Ref: WordNet Search - 3.1. \url{http://wordnetweb.princeton.edu/perl/webwn?s=cleric\&sub=Search+WordNet\&o2=\&o0=1\&o8=1\&o1=1\&o7=\&o5=\&o9=\&o6=\&o3=\&o4=\&h}=. Accessed 18 July 2021.

\item Word: disseminate

Definition: dis·sem·i·nate | dəˈseməˌnāt | verb [with object ] spread (something, especially information) widely: health authorities should foster good practice by disseminating information.

Ref: WordNet Search - 3.1. \url{http://wordnetweb.princeton.edu/perl/webwn?s=disseminate\&sub=Search+WordNet\&o2=\&o0=1\&o8=1\&o1=1\&o7=\&o5=\&o9=\&o6=\&o3=\&o4=\&h}=. Accessed 23 July 2021.

\item Word: encode

 Definition: en·code | inˈkōd, enˈkōd | verb [with object ] convert into a
coded form: using this technique makes it possible to encode and transmit recorded video information.

Ref: WordNet Search - 3.1. \url{http://wordnetweb.princeton.edu/perl/webwn?s=encode\&sub=Search+WordNet\&o2=\&o0=1\&o8=1\&o1=1\&o7=\&o5=\&o9=\&o6=\&o3=\&o4=\&h=0000000}. Accessed 23 July 2021.

\item Word: RFID

 Definition: RFID \textbf{(abbreviation)} radio frequency identification, denoting
technologies that use radio waves to identify people or objects carrying encoded microchips.

Ref: WordNet Search - 3.1. \url{http://wordnetweb.princeton.edu/perl/webwn?s=RFID\&sub=Search+WordNet\&o2=\&o0=1\&o8=1\&o1=1\&o7=\&o5=\&o9=\&o6=\&o3=\&o4=\&h=0000000}. Accessed 23 July 2021.

\item Word: exonerate

Definition:  ex·on·er·ate | iɡˈzänəˌrāt | verb [with object ] 1 (especially of an official body) absolve (someone) from blame for a fault or wrongdoing, especially after due consideration of the case: they should exonerate these men from this crime

Ref: WordNet Search - 3.1. \url{http://wordnetweb.princeton.edu/perl/webwn?s=exonerate\&sub=Search+WordNet\&o2=\&o0=1\&o8=1\&o1=1\&o7=\&o5=\&o9=\&o6=\&o3=\&o4=\&h=0000000}. Accessed 23 July 2021.

\item Word: discourse

 Definition: noun | ˈdisˌkôrs | written or spoken communication or debate: an imagined discourse between two people
traveling in France.

Ref: WordNet Search - 3.1. \url{http://wordnetweb.princeton.edu/perl/webwn?s=discourse\&sub=Search+WordNet\&o2=\&o0=1\&o8=1\&o1=1\&o7=\&o5=\&o9=\&o6=\&o3=\&o4=\&h=0000000}. Accessed 23 July 2021.

\item Word: profilerate

Definition: (v)  \textbf{profilerate} increase rapidly in numbers; multiply.

Ref: WordNet Search - 3.1.
\url{http://wordnetweb.princeton.edu/perl/webwn?s=profilerate\&sub=Search+WordNet\&o2=\&o0=1\&o8=1\&o1=1\&o7=\&o5=\&o9=\&o6=\&o3=\&o4=\&h=0000000}. Accessed 23 July 2021.

\item Word: prodigious

Definition: pro·di·gious | prəˈdijəs | adjective, remarkably or impressively great in extent, size, or degree: the stove consumed a prodigious amount of fuel.

Ref: WordNet Search - 3.1.
\url{http://wordnetweb.princeton.edu/perl/webwn?s=prodigious\&sub=Search+WordNet\&o2=\&o0=1\&o8=1\&o1=1\&o7=\&o5=\&o9=\&o6=\&o3=\&o4=\&h=0000000}. Accessed 23 July 2021.

\item Word: clairvoyant

Definition: clair·voy·ant | ˌklerˈvoiənt | noun a person who claims to have a supernatural ability to perceive events in the future or beyond normal sensory contact

Ref: WordNet Search - 3.1.
\url{http://wordnetweb.princeton.edu/perl/webwn?s=clairvoyant\&sub=Search+WordNet\&o2=\&o0=1\&o8=1\&o1=1\&o7=\&o5=\&o9=\&o6=\&o3=\&o4=\&h=0000000}. Accessed 23 July 2021.
\end{enumerate}

\subsection{Ideas:}
\label{sec:org177e354}
\begin{enumerate}
\item The notion of privacy has become fuzzier at the same time as the
secrecy-enhancing technology of encryption has become widespread (Page 21)
\item His car had a black box-an EDR, that captured every detail about what was
going on before the crash (page 27)
\item Bits mediate our daily lives. It is almost as hard to avoid leaving digital
footprints as it is to avoid touching the ground when we walk
\item ``There is no patient confidentiality'' said Dr. Joseph Heyman. ``It's gone''
\end{enumerate}

\subsection{Journal Entry}
\label{sec:org971416b}
\begin{enumerate}
\item I agree with the notion that privacy has become fuzzier over time. As
encryption and security technologies are becoming more widespread, it seems
people are caring less and less about their privacy when really they should
be caring more. Companies give us a false sense of privacy, when really they
are breaching it more than ever.

The greatest example of this is google. When you open up www.google.com, you
can see multiple mentions of privacy. In reality, google is notorious for
using user information to target ads and search results. They have multiple
analytics and adsense services that companies can purchase.

\item I agree with the idea of having tracking devices in cars. If most people are
given the choice between getting an expensive ticket and facing criminal
charges or lying, most people will choose to lie. Devices like the EDR ensure
we can make a conclusion based on actual data rather than from the victims
point of view.

However, similar to the issue with cellphone locations  the key term is \emph{as
long as that information is kept private and in the hands of the government}.
Private companies shouldn't be able to get the locations of users and use it
when it isn't needed. Examples of this can be determining how to price
billboard advertising, requiring cars to be serviced ever \emph{x} miles.  There can
be certain exceptions (e.g. An insurance company trying to determine who is
at fault), but for the most part this information should be for the
government, and even then only for when the government absolutely requires it

\item Its true that now its extremely difficult to do anything without leaving
digital traces everywhere. I personally think this issue is linked to idea
\#1, people value convenience over privacy. Companies create a false sense of
privacy, and justify all the analytics with improved convenience.

Most people don't want to put effort into maintaining their privacy, or
resist changes to their workflow and life, and unless you make privacy more
convenient, that won't change.

\item Patient Confidentiality is another issue that's increased over the past few
years. Your insurence company knows more than your entire medical history,
and can make descisions on it. Is it fair to offer higher insurence rates to
those who are more likely to fall ill, or does everyone deserve the same
healthcare? In my opinion, targetting those who need good healthcare the most
should be illegal, everyone should have the same healthcare regardless of
their status.
\end{enumerate}

\section{Ghosts in the Machine}
\label{sec:orgce3836e}
\subsection{Vocabulary}
\label{sec:orgb3f87f5}
\begin{enumerate}
\item Word: metadata

Definition: noun  Data that describes other data, as in describing the origin, structure, or characteristics of computer files,

Ref: WordNet Search - 3.1. \url{http://wordnetweb.princeton.edu/perl/webwn?s=metadata\&sub=Search+WordNet\&o2=\&o0=1\&o8=1\&o1=1\&o7=\&o5=\&o9=\&o6=\&o3=\&o4=\&h}=. Accessed 18 July 2021.

\item Word: open source software

Definition: Programs for which the source code is freely available and freely redistributable, with no commercial strings attached.

Ref: WordNet Search - 3.1. \url{http://wordnetweb.princeton.edu/perl/webwn?s=open-source-software\&sub=Search+WordNet\&o2=\&o0=1\&o8=1\&o1=1\&o7=\&o5=\&o9=\&o6=\&o3=\&o4=\&h}=. Accessed 18 July 2021.

\item Word: ascii

Definition: noun  (computer science) a code for information exchange between computers made by different companies; a string of 7 binary

Ref: WordNet Search - 3.1. \url{http://wordnetweb.princeton.edu/perl/webwn?s=ascii\&sub=Search+WordNet\&o2=\&o0=1\&o8=1\&o1=1\&o7=\&o5=\&o9=\&o6=\&o3=\&o4=\&h}=. Accessed 23 July 2021.

\item Word: steganography

Definition:  noun   The deliberate concealment of data within other data, as by embedding digitized text in a digitized image.

Ref: WordNet Search - 3.1. \url{http://wordnetweb.princeton.edu/perl/webwn?s=steganography\&sub=Search+WordNet\&o2=\&o0=1\&o8=1\&o1=1\&o7=\&o5=\&o9=\&o6=\&o3=\&o4=\&h=0000000}. Accessed 23 July 2021.

\item Word: blocks

 Definition: noun (Computer Science) a block is a segment of a large area
that can be used to assign data

Ref: WordNet Search - 3.1. \url{http://wordnetweb.princeton.edu/perl/webwn?s=block\&sub=Search+WordNet\&o2=\&o0=1\&o8=1\&o1=1\&o7=\&o5=\&o9=\&o6=\&o3=\&o4=\&h=0000000}. Accessed 23 July 2021.

\item Word: algorithm

Definition:  noun   A finite set of unambiguous instructions that, given some set of initial conditions, can be performed in a prescribed

Ref: WordNet Search - 3.1. \url{http://wordnetweb.princeton.edu/perl/webwn?s=algorithm\&sub=Search+WordNet\&o2=\&o0=1\&o8=1\&o1=1\&o7=\&o5=\&o9=\&o6=\&o3=\&o4=\&h=0000000}. Accessed 23 July 2021.

\item Word: pixel

Definition:  noun   One of the tiny dots that make up the representation of an image in a computer's memory.

Ref: WordNet Search - 3.1. \url{http://wordnetweb.princeton.edu/perl/webwn?s=pixel\&sub=Search+WordNet\&o2=\&o0=1\&o8=1\&o1=1\&o7=\&o5=\&o9=\&o6=\&o3=\&o4=\&h=0000000}. Accessed 23 July 2021.

\item Word: raster

Definition: (v)  noun  A bitmap image, consisting of a grid of pixels, stored as a sequence of lines.

Ref: WordNet Search - 3.1. \url{http://wordnetweb.princeton.edu/perl/webwn?s=raster\&sub=Search+WordNet\&o2=\&o0=1\&o8=1\&o1=1\&o7=\&o5=\&o9=\&o6=\&o3=\&o4=\&h=0000000}. Accessed 23 July 2021.

\item Word: render

Definition: transitive verb (Computers)  To convert (graphics) from a file into visual form, as on a video display.

Ref: WordNet Search - 3.1. \url{http://wordnetweb.princeton.edu/perl/webwn?s=render\&sub=Search+WordNet\&o2=\&o0=1\&o8=1\&o1=1\&o7=\&o5=\&o9=\&o6=\&o3=\&o4=\&h=0000000}. Accessed 23 July 2021.

\item Word: spam

Definition:  noun   Unsolicited e-mail, often of a commercial nature, sent indiscriminately to multiple mailing lists, individuals, or

Ref: WordNet Search - 3.1.
\end{enumerate}
\url{http://wordnetweb.princeton.edu/perl/webwn?s=spam\&sub=Search+WordNet\&o2=\&o0=1\&o8=1\&o1=1\&o7=\&o5=\&o9=\&o6=\&o3=\&o4=\&h=0000000}. Accessed 23 July 2021.

\subsection{Ideas:}
\label{sec:org7e713e2}
\begin{enumerate}
\item Metadata can help or refute claims (Page 78)
\item With some clever programming, the process could be made unnoticeable, but so
far neither Microsoft nor Apple has made the necessary software investment (Page 102)
\item Free software is a matter of the users freedom to run, copy, distribute,
study, change, and improve the software (page 94)
\item If google holds your documents, they are accessible from anywhere the
internet reaches
\end{enumerate}

\subsection{Journal Entry}
\label{sec:orgf8d9825}
\begin{enumerate}
\item As the text after that statement mentioned,  metadata can be easily altered
to match a criminals statement. Although most people won't know about
metadata and how to alter it, this makes it ineffective for most purposes.

Personally, I think we should remove most metadata. It ends up doing more
harm than good, especially in the case of images, where metadata can help
trace the location where the image was taken. In the case of documents, it
provides easily forge-able data that serves no purpose.

\item In my opinion, there isn't a need to zero all abandoned blocks by default.
This is for several reasons. Firstly, the majority of people have no
intention to sell their storage. The majority of laptops nowadays have
storage soldered down, which would make those drives impossible to resell.
Secondly, those who care about their privacy will likely have other methods
to zero their storage anyways.

Additionally, making this zero-ing behavior the default will bring
performance implications as mentioned in the textbook. It's a slow process
writing to all of the abandoned blocks. The author mentions this issue could
be solved ``with some clever programming,'' but programming isn't magic, in the
end to properly erase these blocks you will have to write to then. A better
solution would be to incorporate this behavior into the filesystem itself,
prioritizing writing to abandoned blocks before free ones. This system
increases drive longevity, reduces the chance of abandoned blocks, and is
already incorporated in many filesystems today

\item I support the Free and Open Source movement. Open source software gives power
to the users. With the open source model come better security, more
features, and more support for users. Compared to commercial (paid) software,
users can not only identify and report bugs, but also fix those bugs
themselves. Since the source code for open source applications is available,
you can clone the repository, change the insecure code, and submit your
changes.  On the other hand, with commercial software, you will have to wait
until the company updates the application.

Additionally, you can add features to open source software. If Microsoft Word is
missing a feature, you can only request Microsoft to add it. However, if
there is an open source editor missing a feature, you can simply fork it and
add the feature yourself. Likewise, most open source editors will never die,
since there is always a community of people working on it, whereas most
commercial software is dependant on the future of the company producing it.

However, there are also drawbacks with open source software. Since there is
no financial backing for most projects, maintainers don't have an incentive
to keep working. Similarly, since coders aren't paid the quality work may not
be as good as that of a commercial project.

\item I support the idea of using cloud-based storage solutions. As of 2020 50\% of all
corporate data is stored in the cloud, up from 30\% just 5 years earlier. As
more and more people start using cloud services,  we should focus on make
data transfer to and from those drives quicker and more secure. Otherwise,
cloud storage is superior to offline storage except for solutions which
require speed (e.g. boot drives) or high security.
\end{enumerate}

\section{Needles in the Haystack}
\label{sec:org141bb07}
\subsection{Vocabulary}
\label{sec:orgac2b55d}
\begin{enumerate}
\item Word: bot

Definition: noun   A software program, such as a spider, that performs automated tasks on the Internet.

Ref: WordNet Search - 3.1. \url{http://wordnetweb.princeton.edu/perl/webwn?s=metadata\&sub=Search+WordNet\&o2=\&o0=1\&o8=1\&o1=1\&o7=\&o5=\&o9=\&o6=\&o3=\&o4=\&h}=. Accessed 18 July 2021.

\item Word: cache

Definition: noun   A fast storage buffer in the central processing unit of a computer.

Ref: WordNet Search - 3.1. \url{http://wordnetweb.princeton.edu/perl/webwn?s=open-source-software\&sub=Search+WordNet\&o2=\&o0=1\&o8=1\&o1=1\&o7=\&o5=\&o9=\&o6=\&o3=\&o4=\&h}=. Accessed 18 July 2021.

\item Word: HTML

Definition: The HyperText Markup Language, or HTML is the standard markup language for documents designed to be displayed in a web browser. It can be assisted by technologies such as Cascading Style Sheets and scripting languages such as JavaScript

Ref: WordNet Search - 3.1. \url{http://wordnetweb.princeton.edu/perl/webwn?s=ascii\&sub=Search+WordNet\&o2=\&o0=1\&o8=1\&o1=1\&o7=\&o5=\&o9=\&o6=\&o3=\&o4=\&h}=. Accessed 23 July 2021.

\item Word: URL

 Definition: the address of a resource (such as a document or website) on the
Internet that consists of a communications protocol followed by the name or
address of a computer on the network and that often includes additional
locating information (such as directory and file names)

Ref: WordNet Search - 3.1. \url{http://wordnetweb.princeton.edu/perl/webwn?s=steganography\&sub=Search+WordNet\&o2=\&o0=1\&o8=1\&o1=1\&o7=\&o5=\&o9=\&o6=\&o3=\&o4=\&h=0000000}. Accessed 23 July 2021.

\item Word: firewall

Definition: noun Computers  A software program or hardware device that restricts communication between a private network or computer system and outside networks.

Ref: WordNet Search - 3.1. \url{http://wordnetweb.princeton.edu/perl/webwn?s=block\&sub=Search+WordNet\&o2=\&o0=1\&o8=1\&o1=1\&o7=\&o5=\&o9=\&o6=\&o3=\&o4=\&h=0000000}. Accessed 23 July 2021.

\item Word: Web 2.0

Definition:  Web 2.0 refers to websites that emphasize user-generated content, ease of use, participatory culture and interoperability for end users.

Ref: WordNet Search - 3.1. \url{http://wordnetweb.princeton.edu/perl/webwn?s=algorithm\&sub=Search+WordNet\&o2=\&o0=1\&o8=1\&o1=1\&o7=\&o5=\&o9=\&o6=\&o3=\&o4=\&h=0000000}. Accessed 23 July 2021.

\item Word: index

Definition:   noun Computers  A list of keywords associated with a record or document, used especially as an aid in searching for information.

Ref: WordNet Search - 3.1. \url{http://wordnetweb.princeton.edu/perl/webwn?s=pixel\&sub=Search+WordNet\&o2=\&o0=1\&o8=1\&o1=1\&o7=\&o5=\&o9=\&o6=\&o3=\&o4=\&h=0000000}. Accessed 23 July 2021.

\item Word: background

Definition: noun Computers  The environment in which programs operate that the user does not engage with directly.

Ref: WordNet Search - 3.1. \url{http://wordnetweb.princeton.edu/perl/webwn?s=raster\&sub=Search+WordNet\&o2=\&o0=1\&o8=1\&o1=1\&o7=\&o5=\&o9=\&o6=\&o3=\&o4=\&h=0000000}. Accessed 23 July 2021.

\item Word: foreground

Definition: noun computing  the application the user is currently interacting with; the application window that appears in front of all others.

Ref: WordNet Search - 3.1. \url{http://wordnetweb.princeton.edu/perl/webwn?s=render\&sub=Search+WordNet\&o2=\&o0=1\&o8=1\&o1=1\&o7=\&o5=\&o9=\&o6=\&o3=\&o4=\&h=0000000}. Accessed 23 July 2021.

\item Word: spider/web crawler

Definition:  noun   A program that automatically retrieves webpages and follows the links on them to retrieve more webpages. Spiders are used by search engines to retrieve publicly accessible webpages for indexing, and they can also be used to check for links to webpages that no longer exist.

Ref: WordNet Search - 3.1. \url{http://wordnetweb.princeton.edu/perl/webwn?s=spam\&sub=Search+WordNet\&o2=\&o0=1\&o8=1\&o1=1\&o7=\&o5=\&o9=\&o6=\&o3=\&o4=\&h=0000000}. Accessed 23 July 2021.
\end{enumerate}

\subsection{Ideas:}
\label{sec:orgaa87f85}
\begin{enumerate}
\item If you try another search engine instead of Google, you'll get different
results. Which is right? Which is better? Which is more authoritative? (page 114)
\item Web 2.0 sites—Facebook and Wikipedia, for example—exploit what economists
call “network effects.” (Page 110)
\item Free software is a matter of the users freedom to run, copy, distribute,
study, change, and improve the software (page 94)
\item If google holds your documents, they are accessible from anywhere the
internet reaches
\end{enumerate}

\subsection{Journal Entry}
\label{sec:org55898a9}
\begin{enumerate}
\item The reason google is so ahead in the search engine space is because of data.
Google is known to collect enormous amounts of data on their users, and this
is both a good and a bad thing. Although there are many privacy issues
surrounding the use of this data, google can use this data to predict what
each user likes, and can also tune the search engine's results to match that
users preference.

Other search engines such as DDG can't provide you ``accurate'' search engine
results, because they don't have the data google has. Neither result is more
accurate, or better, just calculated differently. If you were to disable
google's analytics, log out of gmail, and open an incognito tab, chances are
you would get different results than the ones you would get when logged in.

\item I dislike the shift to Web 2.0 sites, and specifically the excessive use of
Javascript. Javascript originally started off as a scripting language, but
with the V8 engine, was performant enough to be usable elsewhere. Suddenly.
everyone started using Javascript, even when it wasn't required. While there
are benefits to using Javascript, its still quite slow and in most cases can
be substituted for Html5 + CSS

Static, read-only sites (Web 1.0) get a bad reputation for their bland look,
and their issues with ease-of-use. However, css solved many of these issues,
and you can still design user-friendly Web 1.0 sites, which perform much
better than their Web 2.0 counterparts. Web 1.0 sites also have the benefits and
drawbacks of being read-only. For websites like articles, that means there
are no comment sections or request popups. Lastly, javascript isn't a bad
language by nature, people are just using it for the wrong purpose. When used
tastefully, it can still vastly improve websites.

\item My favorite part of free software is that users can improve the software as
they please. If a piece of software that someone like, but doesn't have a
feature they need, the user is free to modify the software as they please,
and submit they're code upstream (back to the source of the software) if they
wish to. On the other hand, with commercial (paid) software, you are stuck
with requesting the company to update the application.

However, there are also drawbacks with open source software. Since there is
no financial backing for most projects, maintainers don't have an incentive
to keep working. Similarly, since coders aren't paid the quality work may not
be as good as that of a commercial project.

\item This is why I like cloud storage over local storage. With local storage you
have to deal with organizing files, keeping storage down, and most
importantly you have to deal with sharing files with others, and making sure
your assignments are accessible elsewhere when you need them. Additionally,
when you share files with others, you need to keep track of changes others do

With google drive, everything is instantly organized and searchable. You
don't have to deal much with keeping storage down, as cloud storage is
extremely cheap. Most importantly, all your files are available as long as
you have an internet connection, and collaborative work is much easier, as
google keeps track of changes, and everything is in one document
\end{enumerate}

\section{Secret Bits}
\label{sec:org4ce30f6}
\subsection{Vocabulary}
\label{sec:org8c9f33a}
\begin{enumerate}
\item Word: Cipher

Definition: noun   a secret or disguised way of writing; a code

Ref: WordNet Search - 3.1. \url{http://wordnetweb.princeton.edu/perl/webwn?s=metadata\&sub=Search+WordNet\&o2=\&o0=1\&o8=1\&o1=1\&o7=\&o5=\&o9=\&o6=\&o3=\&o4=\&h}=. Accessed 18 July 2021.

\item Word: decryption

Definition: noun a text that has been decoded

Ref: WordNet Search - 3.1. \url{http://wordnetweb.princeton.edu/perl/webwn?s=open-source-software\&sub=Search+WordNet\&o2=\&o0=1\&o8=1\&o1=1\&o7=\&o5=\&o9=\&o6=\&o3=\&o4=\&h}=. Accessed 18 July 2021.

\item Word: encryption

Definition: the process of converting information or data into a code, especially to prevent unauthorized access.

Ref: WordNet Search - 3.1. \url{http://wordnetweb.princeton.edu/perl/webwn?s=ascii\&sub=Search+WordNet\&o2=\&o0=1\&o8=1\&o1=1\&o7=\&o5=\&o9=\&o6=\&o3=\&o4=\&h}=. Accessed 23 July 2021.

\item Word: packet

Definition: a block of data transmitted across a network.

Ref: WordNet Search - 3.1. \url{http://wordnetweb.princeton.edu/perl/webwn?s=steganography\&sub=Search+WordNet\&o2=\&o0=1\&o8=1\&o1=1\&o7=\&o5=\&o9=\&o6=\&o3=\&o4=\&h=0000000}. Accessed 23 July 2021.

\item Word: router

Definition: a device which forwards data packets to the appropriate parts of a computer network.

Ref: WordNet Search - 3.1. \url{http://wordnetweb.princeton.edu/perl/webwn?s=block\&sub=Search+WordNet\&o2=\&o0=1\&o8=1\&o1=1\&o7=\&o5=\&o9=\&o6=\&o3=\&o4=\&h=0000000}. Accessed 23 July 2021.

\item Word: plain text

Definition:  text that is not computationally tagged, specially formatted, or written in code.

Ref: WordNet Search - 3.1. \url{http://wordnetweb.princeton.edu/perl/webwn?s=algorithm\&sub=Search+WordNet\&o2=\&o0=1\&o8=1\&o1=1\&o7=\&o5=\&o9=\&o6=\&o3=\&o4=\&h=0000000}. Accessed 23 July 2021.

\item Word: RSA

Definition:   An algorithm used in public key cryptography, commonly used in various protocols for secure online transmission of data.

Ref: WordNet Search - 3.1. \url{http://wordnetweb.princeton.edu/perl/webwn?s=pixel\&sub=Search+WordNet\&o2=\&o0=1\&o8=1\&o1=1\&o7=\&o5=\&o9=\&o6=\&o3=\&o4=\&h=0000000}. Accessed 23 July 2021.

\item Word: private key

Definition: a cryptographic key that can be obtained and used by anyone to encrypt messages intended for a particular recipient, such that the encrypted messages can be deciphered only by using a second key that is known only to the recipient (the private key ).

Ref: WordNet Search - 3.1. \url{http://wordnetweb.princeton.edu/perl/webwn?s=raster\&sub=Search+WordNet\&o2=\&o0=1\&o8=1\&o1=1\&o7=\&o5=\&o9=\&o6=\&o3=\&o4=\&h=0000000}. Accessed 23 July 2021.

\item Word: digital signature

Definition: a type of electronic signature that encrypts documents with digital codes that are particularly difficult to duplicate.

Ref: WordNet Search - 3.1. \url{http://wordnetweb.princeton.edu/perl/webwn?s=render\&sub=Search+WordNet\&o2=\&o0=1\&o8=1\&o1=1\&o7=\&o5=\&o9=\&o6=\&o3=\&o4=\&h=0000000}. Accessed 23 July 2021.

\item Word: one-time pad

Definition:  In cryptography, the one-time pad (OTP) is an encryption technique that cannot be cracked, but requires the use of a single-use pre-shared key that is no smaller than the message being sent.

Ref: WordNet Search - 3.1. \url{http://wordnetweb.princeton.edu/perl/webwn?s=spam\&sub=Search+WordNet\&o2=\&o0=1\&o8=1\&o1=1\&o7=\&o5=\&o9=\&o6=\&o3=\&o4=\&h=0000000}. Accessed 23 July 2021.
\end{enumerate}

\subsection{Ideas:}
\label{sec:org36b2bd0}
\begin{enumerate}
\item Breakthroughs Happen, but News Travels Slowly (Page 175)
\item Having a good system doesn't mean people will use it (Page 175)
\item Today, electronic mail is gradually replacing conventional paper mail, and is
soon to be the norm for everyone, not the novelty it is today (page 189)
\item Overall, the public seems unconcerned about the privacy of communication
today, and the privacy fervor that permeated the crypto wards a decade ago is
nowhere to be seen
\end{enumerate}

\subsection{Journal Entry}
\label{sec:orgcd77b58}
\begin{enumerate}
\item This is an extremely important issue, especially in the software industry.
People are naturally resistant to change, and companies dislike disrupting
their workflow. Because of this, you often see companies and people usingl
decade old software because it still works. The issue isn't creating new
technology, its getting people to switch on it.

Especially with so much new tech quickly popping up and fading away just as
fast as it came, it's easy to see why people are reluctant to change. There's
always a lingering fear that the tool you currently use may stop support the
next day.

\item I agree with this statement. The largest issue with creating new technologies
is that many people are reluctant to switch from the old ones. Every new idea
seems to follow the same cycle. A new technology is created, adopted, and
then either replaced by a newer one or deprecated and left unmaintained.

The issue is when you create a new technology, no one depends on it. You are
free to modify it to your liking. After others start to adopt a new
technology, you can't create breaking changes. This limitation creates a lack
of innovation, until another technology comes along and replaces it. With so
many new technologies popping up and fading away, sometimes within the same
year, it makes sense if people are more reluctant to switch.

\item This statement aged very well. Over time, email has replaced almost all
physical mail,  and is used for communication around the world. Not only is
it secure, but its dependable, and a standard format that works just as well
with ancient technology and modern web browsers. Email's biggest benefit and
drawback is its standardized tried-and-true method of sending information.
While this means that you use several different email clients which all work
perfectly with each other, it also becomes impractical to use email for
things like instant messaging.

\item Its true that now its extremely difficult to do anything without leaving
digital traces everywhere. I personally think this issue is linked to idea
\#1, people value convenience over privacy. Companies create a false sense of
privacy, and justify all the analytics with improved convenience.

Most people don't want to put effort into maintaining their privacy, or
resist changes to their workflow and life, and unless you make privacy more
convenient, that won't change.
\end{enumerate}

\section{Balance Toppled}
\label{sec:orgadbb014}
\subsection{Vocabulary}
\label{sec:org63e66d6}
\begin{enumerate}
\item Word: driftnet

Definition: noun Driftnet is a program which listens to network traffic and picks out images from TCP streams it observes.

Ref: WordNet Search - 3.1. \url{http://wordnetweb.princeton.edu/perl/webwn?s=metadata\&sub=Search+WordNet\&o2=\&o0=1\&o8=1\&o1=1\&o7=\&o5=\&o9=\&o6=\&o3=\&o4=\&h}=. Accessed 18 July 2021.

\item Word: copyright infringement

Definition: noun The unauthorized use of copyrighted material in a manner that violates one of the copyright owner's exclusive rights, such as the right to reproduce or perform the copyrighted work, or to make derivative works that build upon it.

Ref: WordNet Search - 3.1. \url{http://wordnetweb.princeton.edu/perl/webwn?s=open-source-software\&sub=Search+WordNet\&o2=\&o0=1\&o8=1\&o1=1\&o7=\&o5=\&o9=\&o6=\&o3=\&o4=\&h}=. Accessed 18 July 2021.

\item Word: centralized systems

Definition: A system where decisions for the system goal are made in a central mechanism and transmitted to executive components

Ref: WordNet Search - 3.1. \url{http://wordnetweb.princeton.edu/perl/webwn?s=ascii\&sub=Search+WordNet\&o2=\&o0=1\&o8=1\&o1=1\&o7=\&o5=\&o9=\&o6=\&o3=\&o4=\&h}=. Accessed 23 July 2021.

\item Word: DRAM

Definition: DRAM is a type of RAM that stores each bit of data on a separate capacitor. This is an efficient way to store data in memory, because it requires less physical space to store the same amount of data than if it was stored statically.

Ref: WordNet Search - 3.1. \url{http://wordnetweb.princeton.edu/perl/webwn?s=steganography\&sub=Search+WordNet\&o2=\&o0=1\&o8=1\&o1=1\&o7=\&o5=\&o9=\&o6=\&o3=\&o4=\&h=0000000}. Accessed 23 July 2021.

\item Word: digital rights management (DRM)

Definition: Digital rights management tools or technological protection measures are a set of access control technologies for restricting the use of proprietary hardware and copyrighted works. DRM technologies try to control the use, modification, and distribution of copyrighted works, as well as systems within devices that enforce these policies.

Ref: WordNet Search - 3.1. \url{http://wordnetweb.princeton.edu/perl/webwn?s=block\&sub=Search+WordNet\&o2=\&o0=1\&o8=1\&o1=1\&o7=\&o5=\&o9=\&o6=\&o3=\&o4=\&h=0000000}. Accessed 23 July 2021.

\item Word: DMCA

Definition:  The Digital Millennium Copyright Act is a 1998 United States copyright law that implements two 1996 treaties of the World Intellectual Property Organization. It criminalizes production and dissemination of technology, devices, or services intended to circumvent measures that control access to copyrighted works.

Ref: WordNet Search - 3.1. \url{http://wordnetweb.princeton.edu/perl/webwn?s=algorithm\&sub=Search+WordNet\&o2=\&o0=1\&o8=1\&o1=1\&o7=\&o5=\&o9=\&o6=\&o3=\&o4=\&h=0000000}. Accessed 23 July 2021.

\item Word: copyright

Definition: The legal right granted to an author, composer, playwright, publisher, or distributor to exclusive publication, production, sale, or distribution of a literary, musical, dramatic, or artistic work.

Ref: WordNet Search - 3.1. \url{http://wordnetweb.princeton.edu/perl/webwn?s=pixel\&sub=Search+WordNet\&o2=\&o0=1\&o8=1\&o1=1\&o7=\&o5=\&o9=\&o6=\&o3=\&o4=\&h=0000000}. Accessed 23 July 2021.

\item Word: creative commons

Definition: Creative Commons licenses are free copyright licenses that creators can use to indicate how they'd like their work to be used. Creators can choose from a set of six licenses with varying permissions, from the most open license to the least open license.

Ref: WordNet Search - 3.1. \url{http://wordnetweb.princeton.edu/perl/webwn?s=raster\&sub=Search+WordNet\&o2=\&o0=1\&o8=1\&o1=1\&o7=\&o5=\&o9=\&o6=\&o3=\&o4=\&h=0000000}. Accessed 23 July 2021.

\item Word: digital signature

Definition: A digital signature is a mathematical scheme for verifying the authenticity of digital messages or documents.

Ref: WordNet Search - 3.1. \url{http://wordnetweb.princeton.edu/perl/webwn?s=render\&sub=Search+WordNet\&o2=\&o0=1\&o8=1\&o1=1\&o7=\&o5=\&o9=\&o6=\&o3=\&o4=\&h=0000000}. Accessed 23 July 2021.

\item Word: SHA-256

Definition: Secure Hash Algorithm 256, also known as SHA-256, is a one-way function designed to secure digital information. The function uses a complex mathematical process that converts text of any length into 256 bit (64-character long) string of letters and numbers. SHA-256 is called a one-way function since it is not possible (or very difficult) to determine the original input based on the output.

Ref: WordNet Search - 3.1. \url{http://wordnetweb.princeton.edu/perl/webwn?s=spam\&sub=Search+WordNet\&o2=\&o0=1\&o8=1\&o1=1\&o7=\&o5=\&o9=\&o6=\&o3=\&o4=\&h=0000000}. Accessed 23 July 2021.
\end{enumerate}

\subsection{Ideas:}
\label{sec:org38abef6}
\begin{enumerate}
\item The answer is to build a chip into every computer that checks the operating
system each time the system has been modified, the computer will not boot.
(211)
\item A third DRM difficulty is that, in the name of security and virus protection,
we could easily slip into an unwinnable arms race if increase technology
lock-down that provides no real gain for content owners (Page 175)
\item What is wrong is that we have invented the technology to eliminate scarcity,
but we are deliberately throwing it away to benefit those who profit from
scarcity (page 222)
\item Universal has been talking to Sony and other labels about a subscription
service, where users would pay a fixed fee and then get as much music as they
would like. (page 224)
\end{enumerate}

\subsection{Journal Entry}
\label{sec:orgec2241b}
\begin{enumerate}
\item The main purpose of a TPM chip is to generate RSA keys, SHA-256/SHA-1 hashes,
as well as general purpose signatures. However, TPM's don't have a very
strong track record. With the ROCA vulnerability, it was found you could
easily derive a private key from the public key.

That isn't to say the idea is bad. Apples implementation (via the T2 chip,
and later build into their own chip) a 256-bit AES is generated and burnt
into the chip. No one has access to this key, and theoretically no one can
get access to this key. The key is then used to encrypt the storage. The idea
of TPM is good, its just the implementation isn't.

\item This issue with DRM is already occurring, companies are adding more and more
complicated DRM to their products, while providing no real gain for content
owners. Those who want to pirate their content usually find a way to
circumvent this DRM regardless.

In the case of video games, some reviewer
found that a pirated (and therefore DRM-free) version of the game ran 5-6\%
faster than a legally purchased copy. DRM isn't only \emph{not} helping consumers,
its harming them as well.

\item The issue is that companies are still stuck in the 20th century business
model. They produce millions of copies of movies, sell years old music for
over 2\$ each, and then wonder why people want to get it for free. They \emph{need}
that scarcity to survive, because thats the only way their business model
works. As soon as the entertainment industry moves to subscription services
(which has already happened for the most part, e.g. disney+) , most of these
issues with piracy will resolve themselves.

\item I thought it was interesting how this turned out. Only a few years after
that, we saw the launch of music subscriptions (especially apple music and
spotify at the time), which proved that if you give consumers a cheap and
convenient way to enjoy content, the need to pirate content goes away.
\end{enumerate}
\end{document}
