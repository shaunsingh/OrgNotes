% Created 2021-07-29 Thu 09:22
% Intended LaTeX compiler: pdflatex
\documentclass[11pt]{article}
\usepackage[utf8]{inputenc}
\usepackage[T1]{fontenc}
\usepackage{graphicx}
\usepackage{grffile}
\usepackage{longtable}
\usepackage{wrapfig}
\usepackage{rotating}
\usepackage[normalem]{ulem}
\usepackage{amsmath}
\usepackage{textcomp}
\usepackage{amssymb}
\usepackage{capt-of}
\usepackage{hyperref}
\author{Shaurya Singh}
\date{\today}
\title{Ap CSP Summer Assignment}
\hypersetup{
 pdfauthor={Shaurya Singh},
 pdftitle={Ap CSP Summer Assignment},
 pdfkeywords={},
 pdfsubject={},
 pdfcreator={Emacs 28.0.50 (Org mode 9.5)}, 
 pdflang={English}}
\begin{document}

\maketitle
\setcounter{tocdepth}{1}
\tableofcontents


\section{Chapter 1: Digital Explosion}
\label{sec:org9c91384}
\subsection{Vocabulary}
\label{sec:org6eae27f}
\begin{enumerate}
\item Word: bit

Definition: (n) \textbf{bit} (a unit of measurement of information (from binary + digit); the amount of information in a system having two equiprobable states) ``there are 8 bits in a byte''

Ref: WordNet Search - 3.1. \url{http://wordnetweb.princeton.edu/perl/webwn?s=bit\&sub=Search+WordNet\&o2=\&o0=1\&o8=1\&o1=1\&o7=\&o5=\&o9=\&o6=\&o3=\&o4=\&h}=. Accessed 18 July 2021.

\item Word: koan

Definition: (n) \textbf{koan} (a paradoxical anecdote or a riddle that has no solution; used in Zen Buddhism to show the inadequacy of logical reasoning)

Ref: WordNet Search - 3.1. \url{http://wordnetweb.princeton.edu/perl/webwn?s=koan\&sub=Search+WordNet\&o2=\&o0=1\&o8=1\&o1=1\&o7=\&o5=\&o9=\&o6=\&o3=\&o4=\&h}=. Accessed 18 July 2021.

\item Word: ping

Definition: (v) \textbf{ping} (send a message from one computer to another to check whether it is reachable and active) ``ping your machine in the office''

Ref: WordNet Search - 3.1. \url{http://wordnetweb.princeton.edu/perl/webwn?s=Ping\&sub=Search+WordNet\&o2=\&o0=1\&o8=1\&o1=1\&o7=\&o5=\&o9=\&o6=\&o3=\&o4=\&h}=. Accessed 23 July 2021.

\item Word: benign

Definition: (adj) \textbf{benign} (not dangerous to health; not recurrent or progressive (especially of a tumor))

Ref: WordNet Search - 3.1. \url{http://wordnetweb.princeton.edu/perl/webwn?s=benign\&sub=Search+WordNet\&o2=\&o0=1\&o8=1\&o1=1\&o7=\&o5=\&o9=\&o6=\&o3=\&o4=\&h=0000000}. Accessed 23 July 2021.

\item Word: utopian

Definition: (n) \textbf{utopian} (an idealistic (but usually impractical) social reformer) ``a Utopian believes in the ultimate perfectibility of man''

Ref: WordNet Search - 3.1. \url{http://wordnetweb.princeton.edu/perl/webwn?s=utopian\&sub=Search+WordNet\&o2=\&o0=1\&o8=1\&o1=1\&o7=\&o5=\&o9=\&o6=\&o3=\&o4=\&h=0000000}. Accessed 23 July 2021.

\item Word: err

Definition: (v) \textbf{err}, mistake, slip (to make a mistake or be incorrect)

Ref: WordNet Search - 3.1. \url{http://wordnetweb.princeton.edu/perl/webwn?s=err\&sub=Search+WordNet\&o2=\&o0=1\&o8=1\&o1=1\&o7=\&o5=\&o9=\&o6=\&o3=\&o4=\&h=0000000}. Accessed 23 July 2021.

\item Word: paradoxically

Definition: (adv) \textbf{paradoxically} (in a paradoxical manner) ``paradoxically,  ice ages seem to occur when the sun gets hotter''

(adj) \textbf{paradoxical}, self-contradictory (seemingly contradictory but nonetheless possibly true) ``it is paradoxical that standing is more tiring than walking''

Ref: WordNet Search - 3.1. \url{http://wordnetweb.princeton.edu/perl/webwn?s=paradoxically\&sub=Search+WordNet\&o2=\&o0=1\&o8=1\&o1=1\&o7=\&o5=\&o9=\&o6=\&o3=\&o4=\&h=0000000}. Accessed 23 July 2021.

\item Word: expunge

Definition: (v)  \textbf{expunge} (remove by erasing or crossing out or as if by drawing a line)

Ref: WordNet Search - 3.1.
\url{http://wordnetweb.princeton.edu/perl/webwn?s=expunge\&sub=Search+WordNet\&o2=\&o0=1\&o8=1\&o1=1\&o7=\&o5=\&o9=\&o6=\&o3=\&o4=\&h=0000000}. Accessed 23 July 2021.

\item Word: database

Definition: (n) database (an organized body of related information)

Ref: WordNet Search - 3.1.
\url{http://wordnetweb.princeton.edu/perl/webwn?s=database\&sub=Search+WordNet\&o2=\&o0=1\&o8=1\&o1=1\&o7=\&o5=\&o9=\&o6=\&o3=\&o4=\&h=0000000}. Accessed 23 July 2021.

\item Word: blacklist

Definition: (v) \textbf{blacklist} (put on a blacklist so as to banish or cause to be boycotted) ``many books were blacklisted by the Nazis''

Ref: WordNet Search - 3.1.
\url{http://wordnetweb.princeton.edu/perl/webwn?s=blacklist\&sub=Search+WordNet\&o2=\&o0=1\&o8=1\&o1=1\&o7=\&o5=\&o9=\&o6=\&o3=\&o4=\&h=0000000}. Accessed 23 July 2021.
\end{enumerate}

\subsection{Ideas:}
\label{sec:org31ba123}
\begin{enumerate}
\item Companies keep records of cellphone locations
\item ``Its all just bits''
\item ``In fact, processors have hardly grown faster at all''
\item ``By 2011, we may be producing more bits than we can store''
\end{enumerate}

\subsection{Journal Entry}
\label{sec:org81faca7}
\begin{longtable}{|p{2.5cm}|p{6.5cm}|p{1cm}|}
Idea & Opinion & Page\\
\hline
\endfirsthead
\multicolumn{3}{l}{Continued from previous page} \\
\hline

Idea & Opinion & Page \\

\hline
\endhead
\hline\multicolumn{3}{r}{Continued on next page} \\
\endfoot
\endlastfoot
\hline
Companies keep records of cellphone locations & I support the idea of companies keeping track of cellphone locations. This location is extremely useful for finding missing people after emergencies, such as crashing a car in Tanya's case. While the user does lose privacy, the benefits outweigh the drawbacks, and several laws prevent law enforcement from obtaining this private information without a warrant & 1\\
 &  & \\
\hline
``It's all just bits'' & While technically it is all just bits, personally I believe we shouldn't think of it as such. Thinking of computers in terms of bits is like thinking of writing in terms of graphite. While all handwritten work is technically just graphite on paper, its really much more than that. We think of writing in terms of words, phrases, paragraphs, and should think of the computer experience in terms of experiences. Similarly, copyright law is based on text, and so laws pertaining to computers should be based on the computer experience, not what delivers that experience & 5\\
\hline
``In fact, processors have hardly grown faster at all'' & Considering the next sentence is talking about ``multiple processors on the same chip'' I assume this statement is talking about the processor cores themselves. However, data shows that the fastest processors today (AMD Epyc) are about 80 times faster in terms of single core performance, something I wouldn't personally classify as ``hardly.'' Still, its true that processor cores have increased over the years (from single core chips to server chips with over 128 cores in the span of just 20 years). & 8\\
\hline
``By 2011, we may be producing more bits than we can store'' & I found this statement the most interesting in this chapter. We've made leaps and bounds in storage technology since then, nowadays you can buy tens of terabytes of strorage for quite cheap, compared to the expensive ``high end'' 80gb drives you could get in 2008, back when the book released. & 10\\
\end{longtable}
\end{document}
