% Created 2021-09-07 Tue 00:23
% Intended LaTeX compiler: pdflatex
\documentclass{scrartcl}
\usepackage[utf8]{inputenc}
\usepackage[T1]{fontenc}
\usepackage{fontspec}
\usepackage{graphicx}
\usepackage{grffile}
\usepackage{longtable}
\usepackage{wrapfig}
\usepackage{rotating}
\usepackage[normalem]{ulem}
\usepackage{amsmath}
\usepackage{textcomp}
\usepackage{amssymb}
\usepackage{capt-of}
\usepackage[dvipsnames]{xcolor}
\usepackage[colorlinks=true, linkcolor=Blue, citecolor=BrickRed, urlcolor=PineGreen]{hyperref}
\usepackage{indentfirst}
\setmainfont[Ligatures=TeX]{Alegreya}
\setmonofont[Ligatures=TeX]{Liga SFMono Nerd Font}
% features: (acronym par-sep engraved-code-setup engraved-code)
\newcommand{\acr}[1]{\protect\textls*[110]{\scshape #1}}
\newcommand{\acrs}{\protect\scalebox{.91}[.84]{\hspace{0.15ex}s}}
\setlength{\parskip}{\baselineskip}
\setlength{\parindent}{0pt}

\usepackage{fvextra}
\fvset{
  commandchars=\\\{\},
  highlightcolor=white!95!black!80!blue,
  breaklines=true,
  breaksymbol=\color{white!60!black}\tiny\ensuremath{\hookrightarrow}}
\renewcommand\theFancyVerbLine{\footnotesize\color{black!40!white}\arabic{FancyVerbLine}}

\definecolor{codebackground}{HTML}{f7f7f7}
\definecolor{codeborder}{HTML}{f0f0f0}

% TODO have code boxes keep line vertical alignment
\usepackage[breakable,xparse]{tcolorbox}
\DeclareTColorBox[]{Code}{o}%
{colback=codebackground, colframe=codeborder,
  fontupper=\footnotesize,
  colupper=EFD,
  IfNoValueTF={#1}%
  {boxsep=2pt, arc=2.5pt, outer arc=2.5pt,
    boxrule=0.5pt, left=2pt}%
  {boxsep=2.5pt, arc=0pt, outer arc=0pt,
    boxrule=0pt, leftrule=1.5pt, left=0.5pt},
  right=2pt, top=1pt, bottom=0.5pt,
  breakable}

\definecolor{EFD}{HTML}{383a42}
\definecolor{EfD}{HTML}{fafafa}
\newcommand{\EFD}[1]{\colorbox{EfD}{\textcolor{EFD}{#1}}} % default
\definecolor{EFk}{HTML}{e45649}
\newcommand{\EFk}[1]{\textcolor{EFk}{#1}} % font-lock-keyword-face
\definecolor{EFd}{HTML}{84888b}
\newcommand{\EFd}[1]{\textcolor{EFd}{\textit{#1}}} % font-lock-doc-face
\definecolor{EFt}{HTML}{986801}
\newcommand{\EFt}[1]{\textcolor{EFt}{#1}} % font-lock-type-face
\definecolor{EFs}{HTML}{50a14f}
\newcommand{\EFs}[1]{\textcolor{EFs}{#1}} % font-lock-string-face
\definecolor{EFw}{HTML}{986801}
\newcommand{\EFw}[1]{\textcolor{EFw}{#1}} % font-lock-warning-face
\definecolor{EFb}{HTML}{a626a4}
\newcommand{\EFb}[1]{\textcolor{EFb}{#1}} % font-lock-builtin-face
\definecolor{EFct}{HTML}{9ca0a4}
\newcommand{\EFct}[1]{\textcolor{EFct}{#1}} % font-lock-comment-face
\definecolor{EFc}{HTML}{b751b6}
\newcommand{\EFc}[1]{\textcolor{EFc}{#1}} % font-lock-constant-face
\definecolor{EFpp}{HTML}{4078f2}
\newcommand{\EFpp}[1]{\textcolor{EFpp}{\textbf{#1}}} % font-lock-preprocessor-face
\definecolor{EFnc}{HTML}{4078f2}
\newcommand{\EFnc}[1]{\textcolor{EFnc}{\textbf{#1}}} % font-lock-negation-char-face
\definecolor{EFv}{HTML}{6a1868}
\newcommand{\EFv}[1]{\textcolor{EFv}{#1}} % font-lock-variable-name-face
\definecolor{EFf}{HTML}{a626a4}
\newcommand{\EFf}[1]{\textcolor{EFf}{#1}} % font-lock-function-name-face
\definecolor{EFcd}{HTML}{9ca0a4}
\newcommand{\EFcd}[1]{\textcolor{EFcd}{#1}} % font-lock-comment-delimiter-face
\definecolor{EFrc}{HTML}{4078f2}
\newcommand{\EFrc}[1]{\textcolor{EFrc}{\textbf{#1}}} % font-lock-regexp-grouping-construct
\definecolor{EFrb}{HTML}{4078f2}
\newcommand{\EFrb}[1]{\textcolor{EFrb}{\textbf{#1}}} % font-lock-regexp-grouping-backslash
\definecolor{Efob}{HTML}{e7e7e7}
\newcommand{\EFob}[1]{\colorbox{Efob}{#1}} % org-block
\definecolor{EFhn}{HTML}{da8548}
\newcommand{\EFhn}[1]{\textcolor{EFhn}{\textbf{#1}}} % highlight-numbers-number
\definecolor{EFhq}{HTML}{4078f2}
\newcommand{\EFhq}[1]{\textcolor{EFhq}{#1}} % highlight-quoted-quote
\definecolor{EFhs}{HTML}{986801}
\newcommand{\EFhs}[1]{\textcolor{EFhs}{#1}} % highlight-quoted-symbol
\definecolor{EFrdi}{HTML}{4078f2}
\newcommand{\EFrdi}[1]{\textcolor{EFrdi}{#1}} % rainbow-delimiters-depth-1-face
\definecolor{EFrdii}{HTML}{a626a4}
\newcommand{\EFrdii}[1]{\textcolor{EFrdii}{#1}} % rainbow-delimiters-depth-2-face
\definecolor{EFrdiii}{HTML}{50a14f}
\newcommand{\EFrdiii}[1]{\textcolor{EFrdiii}{#1}} % rainbow-delimiters-depth-3-face
\definecolor{EFrdiv}{HTML}{b751b6}
\newcommand{\EFrdiv}[1]{\textcolor{EFrdiv}{#1}} % rainbow-delimiters-depth-4-face
\definecolor{EFrdv}{HTML}{4db5bd}
\newcommand{\EFrdv}[1]{\textcolor{EFrdv}{#1}} % rainbow-delimiters-depth-5-face
\definecolor{EFrdvi}{HTML}{4078f2}
\newcommand{\EFrdvi}[1]{\textcolor{EFrdvi}{#1}} % rainbow-delimiters-depth-6-face
\definecolor{EFrdvii}{HTML}{a626a4}
\newcommand{\EFrdvii}[1]{\textcolor{EFrdvii}{#1}} % rainbow-delimiters-depth-7-face
\definecolor{EFrdiix}{HTML}{50a14f}
\newcommand{\EFrdiix}[1]{\textcolor{EFrdiix}{#1}} % rainbow-delimiters-depth-8-face
\definecolor{EFrdix}{HTML}{b751b6}
\newcommand{\EFrdix}[1]{\textcolor{EFrdix}{#1}} % rainbow-delimiters-depth-9-face
% end features

%% make document follow Emacs theme

\definecolor{obg}{HTML}{fafafa}
\definecolor{ofg}{HTML}{383a42}

\pagecolor{obg}
\color{ofg}

% list labels

\definecolor{itemlabel}{HTML}{4078f2}

\renewcommand{\labelitemi}{\textcolor{itemlabel}{\textbullet}}
\renewcommand{\labelitemii}{\textcolor{itemlabel}{\normalfont\bfseries \textendash}}
\renewcommand{\labelitemiii}{\textcolor{itemlabel}{\textasteriskcentered}}
\renewcommand{\labelitemiv}{\textcolor{itemlabel}{\textperiodcentered}}

\renewcommand{\labelenumi}{\textcolor{itemlabel}{\theenumi.}}
\renewcommand{\labelenumii}{\textcolor{itemlabel}{(\theenumii)}}
\renewcommand{\labelenumiii}{\textcolor{itemlabel}{\theenumiii.}}
\renewcommand{\labelenumiv}{\textcolor{itemlabel}{\theenumiv.}}

% structural elements

\definecolor{documentTitle}{HTML}{a626a4}
\definecolor{documentInfo}{HTML}{a626a4}
\definecolor{level1}{HTML}{e45649}
\definecolor{level2}{HTML}{da8548}
\definecolor{level3}{HTML}{b751b6}
\definecolor{level4}{HTML}{6f99f5}
\definecolor{level5}{HTML}{bc5cba}
\definecolor{level6}{HTML}{9fbbf8}
\definecolor{level7}{HTML}{d292d1}
\definecolor{level8}{HTML}{d8e4fc}

\addtokomafont{title}{\color{documentTitle}}
\addtokomafont{author}{\color{documentInfo}}
\addtokomafont{date}{\color{documentInfo}}
\addtokomafont{section}{\color{level1}}
\newkomafont{sectionprefix}{\color{level1}}
\addtokomafont{subsection}{\color{level2}}
\newkomafont{subsectionprefix}{\color{level2}}
\addtokomafont{subsubsection}{\color{level3}}
\newkomafont{subsubsectionprefix}{\color{level3}}
\addtokomafont{paragraph}{\color{level4}}
\newkomafont{paragraphprefix}{\color{level4}}
\addtokomafont{subparagraph}{\color{level5}}
\newkomafont{subparagraphprefix}{\color{level5}}

% textual elements

\definecolor{link}{HTML}{4078f2}
\definecolor{cite}{HTML}{800080}
\definecolor{itemlabel}{HTML}{4078f2}
\definecolor{code}{HTML}{da8548}
\definecolor{verbatim}{HTML}{50a14f}

\renewcommand{\labelitemi}{\textcolor{itemlabel}{\textbullet}}
\renewcommand{\labelitemii}{\textcolor{itemlabel}{\normalfont\bfseries \textendash}}
\renewcommand{\labelitemiii}{\textcolor{itemlabel}{\textasteriskcentered}}
\renewcommand{\labelitemiv}{\textcolor{itemlabel}{\textperiodcentered}}

\renewcommand{\labelenumi}{\textcolor{itemlabel}{\theenumi.}}
\renewcommand{\labelenumii}{\textcolor{itemlabel}{(\theenumii)}}
\renewcommand{\labelenumiii}{\textcolor{itemlabel}{\theenumiii.}}
\renewcommand{\labelenumiv}{\textcolor{itemlabel}{\theenumiv.}}

\DeclareTextFontCommand{\texttt}{\color{code}\ttfamily}
\makeatletter
\def\verbatim@font{\color{verbatim}\normalfont\ttfamily}
\makeatother

% code blocks

\definecolor{codebackground}{HTML}{fafafa}
\colorlet{EFD}{ofg}
\definecolor{codeborder}{HTML}{f0f0f0}

%% end customisations

\author{Shaurya Singh}
\date{}
\title{CoderZ Summer Assignment \#3}
\colorlet{greenyblue}{blue!70!green}
\colorlet{blueygreen}{blue!40!green}
\providecolor{link}{named}{greenyblue}
\providecolor{cite}{named}{blueygreen}
\hypersetup{
  pdfauthor={Shaurya Singh},
  pdftitle={CoderZ Summer Assignment \#3},
  pdfkeywords={},
  pdfsubject={},
  pdfcreator={Emacs 28.0.50 (Org mode 9.5)},
  pdflang={English},
  breaklinks=true,
  colorlinks=true,
  linkcolor=,
  urlcolor=link,
  citecolor=cite
}
\urlstyle{same}
\begin{document}

\maketitle
\setcounter{tocdepth}{2}
\tableofcontents

Hello everyone! This is the third assignment you will receive this summer. By this
point, all of you should have received and submitted Assignment \#1 as well as
Assignment \#2, which had you learn the basics of python and variables . If you
still need help, feel free to email me at shaunsingh0207@gmail.com.

In Assignment \#3, you will learn how to use
operators, descision making, and loops

\section{Operators}
\label{sec:org5f6c26d}
Operators are constructs which manipulate the value of operands. Operands are
the numbers or values you are acting on, and the operator is what you're doing
to them

Consider the expression 6 + 10 = 16. Here, 6 and 10 are called the operands and + is called the operator.

\subsection{Types of Operator}
\label{sec:org993bed8}
You will need to know the following operators from the python language:
\begin{itemize}
\item Arithmetic Operators
\item Comparison (Relational) Operators
\item Assignment Operators
\item Logical Operators
\end{itemize}

\subsection{Arithmetic Operators}
\label{sec:org4111780}
Arithmetic Operators are the ones you commonly use in math. This consists of
addition, subtraction, multiplication, division, modulus, exponents, and floor
division. An example is below:
\begin{Code}
\begin{Verbatim}[]
\color{EFD}\textcolor[HTML]{9ca0a4}{\#!/usr/bin/python3}

\EFv{a} \textcolor[HTML]{e45649}{=} \textcolor[HTML]{b751b6}{21}
\EFv{b} \textcolor[HTML]{e45649}{=} \textcolor[HTML]{b751b6}{10}
\EFv{c} \textcolor[HTML]{e45649}{=} \textcolor[HTML]{b751b6}{0}

\EFv{c} \textcolor[HTML]{e45649}{=} a \textcolor[HTML]{e45649}{+} b
\textcolor[HTML]{a626a4}{\textbf{print}} (\textcolor[HTML]{50a14f}{"Line 1 - Value of c is "}, c)

\EFv{c} \textcolor[HTML]{e45649}{=} a \textcolor[HTML]{e45649}{-} b
\textcolor[HTML]{a626a4}{\textbf{print}} (\textcolor[HTML]{50a14f}{"Line 2 - Value of c is "}, c )

\EFv{c} \textcolor[HTML]{e45649}{=} a \textcolor[HTML]{e45649}{*} b
\textcolor[HTML]{a626a4}{\textbf{print}} (\textcolor[HTML]{50a14f}{"Line 3 - Value of c is "}, c)

\EFv{c} \textcolor[HTML]{e45649}{=} a \textcolor[HTML]{e45649}{/} b
\textcolor[HTML]{a626a4}{\textbf{print}} (\textcolor[HTML]{50a14f}{"Line 4 - Value of c is "}, c )

\EFv{c} \textcolor[HTML]{e45649}{=} a \textcolor[HTML]{e45649}{\%} b
\textcolor[HTML]{a626a4}{\textbf{print}} (\textcolor[HTML]{50a14f}{"Line 5 - Value of c is "}, c)

\EFv{a} \textcolor[HTML]{e45649}{=} \textcolor[HTML]{b751b6}{2}
\EFv{b} \textcolor[HTML]{e45649}{=} \textcolor[HTML]{b751b6}{3}
\EFv{c} \textcolor[HTML]{e45649}{=} a\textcolor[HTML]{e45649}{**}b
\textcolor[HTML]{a626a4}{\textbf{print}} (\textcolor[HTML]{50a14f}{"Line 6 - Value of c is "}, c)

\EFv{a} \textcolor[HTML]{e45649}{=} \textcolor[HTML]{b751b6}{10}
\EFv{b} \textcolor[HTML]{e45649}{=} \textcolor[HTML]{b751b6}{5}
\EFv{c} \textcolor[HTML]{e45649}{=} a\textcolor[HTML]{e45649}{//}b
\textcolor[HTML]{a626a4}{\textbf{print}} (\textcolor[HTML]{50a14f}{"Line 7 - Value of c is "}, c)
\end{Verbatim}
\end{Code}

\begin{verbatim}
Line 1 - Value of c is  31
Line 2 - Value of c is  11
Line 3 - Value of c is  210
Line 4 - Value of c is  2.1
Line 5 - Value of c is  1
Line 6 - Value of c is  8
Line 7 - Value of c is  2
\end{verbatim}

\subsection{Comparison Operators}
\label{sec:org7165f22}
Comparison operators compare values. These consist of equal, not equal, greater,
less, greater/equal, less/equal. An example is below:

\begin{Code}
\begin{Verbatim}[]
\color{EFD}\textcolor[HTML]{9ca0a4}{\#!/usr/bin/python3}

\EFv{a} \textcolor[HTML]{e45649}{=} \textcolor[HTML]{b751b6}{21}
\EFv{b} \textcolor[HTML]{e45649}{=} \textcolor[HTML]{b751b6}{10}

\textcolor[HTML]{e45649}{if} ( a \textcolor[HTML]{e45649}{==} b ):
   \textcolor[HTML]{a626a4}{\textbf{print}} (\textcolor[HTML]{50a14f}{"Line 1 - a is equal to b"})
\textcolor[HTML]{e45649}{else}:
   \textcolor[HTML]{a626a4}{\textbf{print}} (\textcolor[HTML]{50a14f}{"Line 1 - a is not equal to b"})

\textcolor[HTML]{e45649}{if} ( a \textcolor[HTML]{e45649}{!=} b ):
   \textcolor[HTML]{a626a4}{\textbf{print}} (\textcolor[HTML]{50a14f}{"Line 2 - a is not equal to b"})
\textcolor[HTML]{e45649}{else}:
   \textcolor[HTML]{a626a4}{\textbf{print}} (\textcolor[HTML]{50a14f}{"Line 2 - a is equal to b"})

\textcolor[HTML]{e45649}{if} ( a \textcolor[HTML]{e45649}{<} b ):
   \textcolor[HTML]{a626a4}{\textbf{print}} (\textcolor[HTML]{50a14f}{"Line 3 - a is less than b"} )
\textcolor[HTML]{e45649}{else}:
   \textcolor[HTML]{a626a4}{\textbf{print}} (\textcolor[HTML]{50a14f}{"Line 3 - a is not less than b"})

\textcolor[HTML]{e45649}{if} ( a \textcolor[HTML]{e45649}{>} b ):
   \textcolor[HTML]{a626a4}{\textbf{print}} (\textcolor[HTML]{50a14f}{"Line 4 - a is greater than b"})
\textcolor[HTML]{e45649}{else}:
   \textcolor[HTML]{a626a4}{\textbf{print}} (\textcolor[HTML]{50a14f}{"Line 4 - a is not greater than b"})

\textcolor[HTML]{6a1868}{a},\textcolor[HTML]{6a1868}{b} \textcolor[HTML]{e45649}{=} b,a \textcolor[HTML]{9ca0a4}{\#values of a and b swapped. a becomes 10, b becomes 21}

\textcolor[HTML]{e45649}{if} ( a \textcolor[HTML]{e45649}{<=} b ):
   \textcolor[HTML]{a626a4}{\textbf{print}} (\textcolor[HTML]{50a14f}{"Line 5 - a is either less than or equal to  b"})
\textcolor[HTML]{e45649}{else}:
   \textcolor[HTML]{a626a4}{\textbf{print}} (\textcolor[HTML]{50a14f}{"Line 5 - a is neither less than nor equal to  b"})

\textcolor[HTML]{e45649}{if} ( b \textcolor[HTML]{e45649}{>=} a ):
   \textcolor[HTML]{a626a4}{\textbf{print}} (\textcolor[HTML]{50a14f}{"Line 6 - b is either greater than  or equal to b"})
\textcolor[HTML]{e45649}{else}:
   \textcolor[HTML]{a626a4}{\textbf{print}} (\textcolor[HTML]{50a14f}{"Line 6 - b is neither greater than  nor equal to b"})
\end{Verbatim}
\end{Code}

\begin{verbatim}
Line 1 - a is not equal to b
Line 2 - a is not equal to b
Line 3 - a is not less than b
Line 4 - a is greater than b
Line 5 - a is either less than or equal to  b
Line 6 - b is either greater than  or equal to b
\end{verbatim}

\subsection{Assignment operators}
\label{sec:orgcc2aa28}
Assignment operators assign values. These consist of assign, add and assign,
subtract and assign, multiply and assign, divide and assign, modulus and assign,
exponent and assign. The only one you need to know is assign, the rest are a
combination of the arithmatic operators and assign. An example is below
\begin{Code}
\begin{Verbatim}[]
\color{EFD}\textcolor[HTML]{9ca0a4}{\#!/usr/bin/python3}

\EFv{a} \textcolor[HTML]{e45649}{=} \textcolor[HTML]{b751b6}{21}
\EFv{b} \textcolor[HTML]{e45649}{=} \textcolor[HTML]{b751b6}{10}
\EFv{c} \textcolor[HTML]{e45649}{=} \textcolor[HTML]{b751b6}{0}

\EFv{c} \textcolor[HTML]{e45649}{=} a \textcolor[HTML]{e45649}{+} b
\textcolor[HTML]{a626a4}{\textbf{print}} (\textcolor[HTML]{50a14f}{"Line 1 - Value of c is "}, c)

\EFv{c} \textcolor[HTML]{e45649}{+=} a
\textcolor[HTML]{a626a4}{\textbf{print}} (\textcolor[HTML]{50a14f}{"Line 2 - Value of c is "}, c )

\EFv{c} \textcolor[HTML]{e45649}{*=} a
\textcolor[HTML]{a626a4}{\textbf{print}} (\textcolor[HTML]{50a14f}{"Line 3 - Value of c is "}, c )

\EFv{c} \textcolor[HTML]{e45649}{/=} a
\textcolor[HTML]{a626a4}{\textbf{print}} (\textcolor[HTML]{50a14f}{"Line 4 - Value of c is "}, c )

\EFv{c}  \textcolor[HTML]{e45649}{=} \textcolor[HTML]{b751b6}{2}
\EFv{c} \textcolor[HTML]{e45649}{\%=} a
\textcolor[HTML]{a626a4}{\textbf{print}} (\textcolor[HTML]{50a14f}{"Line 5 - Value of c is "}, c)

\EFv{c} \textcolor[HTML]{e45649}{**=} a
\textcolor[HTML]{a626a4}{\textbf{print}} (\textcolor[HTML]{50a14f}{"Line 6 - Value of c is "}, c)

\EFv{c} \textcolor[HTML]{e45649}{//=} a
\textcolor[HTML]{a626a4}{\textbf{print}} (\textcolor[HTML]{50a14f}{"Line 7 - Value of c is "}, c)
\end{Verbatim}
\end{Code}

\begin{verbatim}
Line 1 - Value of c is  31
Line 2 - Value of c is  52
Line 3 - Value of c is  1092
Line 4 - Value of c is  52.0
Line 5 - Value of c is  2
Line 6 - Value of c is  2097152
Line 7 - Value of c is  99864
\end{verbatim}

\subsection{Logical Operators}
\label{sec:org95c2b49}
A logical operator is a symbol or word used to connect two or more expressions
such that the value of the compound expression produced depends only on that of
the original expressions and on the meaning of the operator. These consist of
AND, OR, and NOT.

\section{Descision making}
\label{sec:orgbb8c222}
Decision-making is the anticipation of conditions occurring during the execution of a program and specified actions taken according to the conditions.

Decision structures evaluate multiple expressions, which produce TRUE or FALSE
as the outcome. You need to determine which action to take and which statements
to execute if the outcome is TRUE or FALSE otherwise

Python assumes any non-zero and non-null values are true, and any zero or null
values as false. Python has the following types of decision making statements,
as do many other languages:

\subsection{If statement}
\label{sec:orga5f8c92}
An if statement consists of a bolean experission followed by one or more
statements. It can execute code depending on the result of the statement
\begin{Code}
\begin{Verbatim}[]
\color{EFD}\textcolor[HTML]{9ca0a4}{\#!/usr/bin/python3}

\EFv{var1} \textcolor[HTML]{e45649}{=} \textcolor[HTML]{b751b6}{100}
\textcolor[HTML]{e45649}{if} var1:
   \textcolor[HTML]{a626a4}{\textbf{print}} (\textcolor[HTML]{50a14f}{"1 - Got a true expression value"})
   \textcolor[HTML]{a626a4}{\textbf{print}} (var1)

\EFv{var2} \textcolor[HTML]{e45649}{=} \textcolor[HTML]{b751b6}{0}
\textcolor[HTML]{e45649}{if} var2:
   \textcolor[HTML]{a626a4}{\textbf{print}} (\textcolor[HTML]{50a14f}{"2 - Got a true expression value"})
   \textcolor[HTML]{a626a4}{\textbf{print}} (var2)
\textcolor[HTML]{a626a4}{\textbf{print}} (\textcolor[HTML]{50a14f}{"Good bye!"})
\end{Verbatim}
\end{Code}

\begin{verbatim}
1 - Got a true expression value
100
Good bye!
\end{verbatim}

\subsection{If else statement}
\label{sec:org45e47ca}
An if statement can also be followed by an else statement, which executes if the
expression returns FALSE
\begin{Code}
\begin{Verbatim}[]
\color{EFD}\textcolor[HTML]{9ca0a4}{\#!/usr/bin/python3}

\EFv{amount} \textcolor[HTML]{e45649}{=} \textcolor[HTML]{a626a4}{\textbf{int}}(\textcolor[HTML]{b751b6}{100})

\textcolor[HTML]{e45649}{if} amount\textcolor[HTML]{e45649}{<}\textcolor[HTML]{b751b6}{1000}:
   \EFv{discount} \textcolor[HTML]{e45649}{=} amount\textcolor[HTML]{e45649}{*}\textcolor[HTML]{b751b6}{0.05}
   \textcolor[HTML]{a626a4}{\textbf{print}} (\textcolor[HTML]{50a14f}{"Discount"},discount)
\textcolor[HTML]{e45649}{else}:
   discount \textcolor[HTML]{e45649}{=} amount\textcolor[HTML]{e45649}{*}\textcolor[HTML]{b751b6}{0.10}
   \textcolor[HTML]{a626a4}{\textbf{print}} (\textcolor[HTML]{50a14f}{"Discount"},discount)

\textcolor[HTML]{a626a4}{\textbf{print}} (\textcolor[HTML]{50a14f}{"Net payable:"},amount\textcolor[HTML]{e45649}{-}discount)
\end{Verbatim}
\end{Code}

\begin{verbatim}
Discount 5.0
Net payable: 95.0
\end{verbatim}

\subsection{Nested if statements}
\label{sec:orgc64f3d8}
You can use if statements (and if else statements) within other if and if else
statements too!
\begin{Code}
\begin{Verbatim}[]
\color{EFD}\textcolor[HTML]{9ca0a4}{\# !/usr/bin/python3}

\EFv{num} \textcolor[HTML]{e45649}{=} \textcolor[HTML]{a626a4}{\textbf{int}}(\textcolor[HTML]{b751b6}{100})
\textcolor[HTML]{e45649}{if} num\textcolor[HTML]{e45649}{\%}\textcolor[HTML]{b751b6}{2} \textcolor[HTML]{e45649}{==} \textcolor[HTML]{b751b6}{0}:
   \textcolor[HTML]{e45649}{if} num\textcolor[HTML]{e45649}{\%}\textcolor[HTML]{b751b6}{3} \textcolor[HTML]{e45649}{==} \textcolor[HTML]{b751b6}{0}:
      \textcolor[HTML]{a626a4}{\textbf{print}} (\textcolor[HTML]{50a14f}{"Divisible by 3 and 2"})
   \textcolor[HTML]{e45649}{else}:
      \textcolor[HTML]{a626a4}{\textbf{print}} (\textcolor[HTML]{50a14f}{"divisible by 2 not divisible by 3"})
\textcolor[HTML]{e45649}{else}:
   \textcolor[HTML]{e45649}{if} num\textcolor[HTML]{e45649}{\%}\textcolor[HTML]{b751b6}{3} \textcolor[HTML]{e45649}{==} \textcolor[HTML]{b751b6}{0}:
      \textcolor[HTML]{a626a4}{\textbf{print}} (\textcolor[HTML]{50a14f}{"divisible by 3 not divisible by 2"})
   \textcolor[HTML]{e45649}{else}:
      \textcolor[HTML]{a626a4}{\textbf{print}}  (\textcolor[HTML]{50a14f}{"not Divisible by 2 not divisible by 3"})
\end{Verbatim}
\end{Code}

\begin{verbatim}
divisible by 2 not divisible by 3
\end{verbatim}

\section{Loops}
\label{sec:org9c33d74}
In general, statements are executed sequentially − The first statement in a function is executed first, followed by the second, and so on. There may be a situation when you need to execute a block of code several number of times.

Programming languages provide various control structures that allow more complicated execution paths.

A loop statement allows us to execute a statement or group of statements
multiple times.

Like If statements, we can also use loops inside other while or for loops!

In python, we have the following types of loops

\subsection{While loop}
\label{sec:orgbcd286f}
While loops repeat a statement or a group of statements while a given condition
is true. An example is
\begin{Code}
\begin{Verbatim}[]
\color{EFD}\textcolor[HTML]{9ca0a4}{\#!/usr/bin/python3}

\EFv{count} \textcolor[HTML]{e45649}{=} \textcolor[HTML]{b751b6}{0}
\textcolor[HTML]{e45649}{while} (count \textcolor[HTML]{e45649}{<} \textcolor[HTML]{b751b6}{4}):
   \textcolor[HTML]{a626a4}{\textbf{print}} (\textcolor[HTML]{50a14f}{'The count is:'}, count)
   count \textcolor[HTML]{e45649}{=} count \textcolor[HTML]{e45649}{+} \textcolor[HTML]{b751b6}{1}

\textcolor[HTML]{a626a4}{\textbf{print}} (\textcolor[HTML]{50a14f}{"Good bye!"})
\end{Verbatim}
\end{Code}

\begin{verbatim}
The count is: 0
The count is: 1
The count is: 2
The count is: 3
Good bye!
\end{verbatim}

\subsection{For loop}
\label{sec:org8103c30}
For loops execute a sequence of statements multiple times and abbreviate the
code that manages the loop variable. An example is traversing a sequence of
strings or a list sequence
\begin{Code}
\begin{Verbatim}[]
\color{EFD}\textcolor[HTML]{9ca0a4}{\#!/usr/bin/python3}

\textcolor[HTML]{e45649}{for} \textcolor[HTML]{6a1868}{letter} \textcolor[HTML]{e45649}{in} \textcolor[HTML]{50a14f}{'Python'}:     \textcolor[HTML]{9ca0a4}{\# traversal of a string sequence}
   \textcolor[HTML]{a626a4}{\textbf{print}} (\textcolor[HTML]{50a14f}{'Current Letter :'}, letter)
\textcolor[HTML]{a626a4}{\textbf{print}}()
fruits \textcolor[HTML]{e45649}{=} [\textcolor[HTML]{50a14f}{'banana'}, \textcolor[HTML]{50a14f}{'apple'},  \textcolor[HTML]{50a14f}{'mango'}]

\textcolor[HTML]{e45649}{for} \textcolor[HTML]{6a1868}{fruit} \textcolor[HTML]{e45649}{in} fruits:        \textcolor[HTML]{9ca0a4}{\# traversal of List sequence}
   \textcolor[HTML]{a626a4}{\textbf{print}} (\textcolor[HTML]{50a14f}{'Current fruit :'}, fruit)

\textcolor[HTML]{a626a4}{\textbf{print}} (\textcolor[HTML]{50a14f}{"Good bye!"})
\end{Verbatim}
\end{Code}

\begin{Code}[alt]
\begin{verbatim}
Current Letter : P
Current Letter : y
Current Letter : t
Current Letter : h
Current Letter : o
Current Letter : n

Current fruit : banana
Current fruit : apple
Current fruit : mango
Good bye!
\end{verbatim}
\end{Code}

\subsection{Loop Control Statements}
\label{sec:org1816013}
Loop control statemenst change the execution of a loop. We can break, continue,
or pass something to a loop. Examples of these are below:

Break: Terminates the loop
\begin{Code}
\begin{Verbatim}[]
\color{EFD}\textcolor[HTML]{9ca0a4}{\#!/usr/bin/python3}

\textcolor[HTML]{e45649}{for} \textcolor[HTML]{6a1868}{letter} \textcolor[HTML]{e45649}{in} \textcolor[HTML]{50a14f}{'Python'}:     \textcolor[HTML]{9ca0a4}{\# First Example}
   \textcolor[HTML]{e45649}{if} letter \textcolor[HTML]{e45649}{==} \textcolor[HTML]{50a14f}{'h'}:
      \textcolor[HTML]{e45649}{break}
   \textcolor[HTML]{a626a4}{\textbf{print}} (\textcolor[HTML]{50a14f}{'Current Letter :'}, letter)

var \textcolor[HTML]{e45649}{=} \textcolor[HTML]{b751b6}{10}                    \textcolor[HTML]{9ca0a4}{\# Second Example}
\textcolor[HTML]{e45649}{while} var \textcolor[HTML]{e45649}{>} \textcolor[HTML]{b751b6}{0}:
   \textcolor[HTML]{a626a4}{\textbf{print}} (\textcolor[HTML]{50a14f}{'Current variable value :'}, var)
   var \textcolor[HTML]{e45649}{=} var \textcolor[HTML]{e45649}{-}\textcolor[HTML]{b751b6}{1}
   \textcolor[HTML]{e45649}{if} var \textcolor[HTML]{e45649}{==} \textcolor[HTML]{b751b6}{5}:
      \textcolor[HTML]{e45649}{break}

\textcolor[HTML]{a626a4}{\textbf{print}} (\textcolor[HTML]{50a14f}{"Good bye!"})
\end{Verbatim}
\end{Code}

\begin{verbatim}
Current Letter : P
Current Letter : y
Current Letter : t
Current variable value : 10
Current variable value : 9
Current variable value : 8
Current variable value : 7
Current variable value : 6
Good bye!
\end{verbatim}


Continue: Causes the loop to skip the rest of its body and retest
\begin{Code}
\begin{Verbatim}[]
\color{EFD}\textcolor[HTML]{9ca0a4}{\#!/usr/bin/python3}

\textcolor[HTML]{e45649}{for} \textcolor[HTML]{6a1868}{letter} \textcolor[HTML]{e45649}{in} \textcolor[HTML]{50a14f}{'Python'}:     \textcolor[HTML]{9ca0a4}{\# First Example}
   \textcolor[HTML]{e45649}{if} letter \textcolor[HTML]{e45649}{==} \textcolor[HTML]{50a14f}{'h'}:
      \textcolor[HTML]{e45649}{continue}
   \textcolor[HTML]{a626a4}{\textbf{print}} (\textcolor[HTML]{50a14f}{'Current Letter :'}, letter)

var \textcolor[HTML]{e45649}{=} \textcolor[HTML]{b751b6}{10}                    \textcolor[HTML]{9ca0a4}{\# Second Example}
\textcolor[HTML]{e45649}{while} var \textcolor[HTML]{e45649}{>} \textcolor[HTML]{b751b6}{0}:
   var \textcolor[HTML]{e45649}{=} var \textcolor[HTML]{e45649}{-}\textcolor[HTML]{b751b6}{1}
   \textcolor[HTML]{e45649}{if} var \textcolor[HTML]{e45649}{==} \textcolor[HTML]{b751b6}{5}:
      \textcolor[HTML]{e45649}{continue}
   \textcolor[HTML]{a626a4}{\textbf{print}} (\textcolor[HTML]{50a14f}{'Current variable value :'}, var)
\textcolor[HTML]{a626a4}{\textbf{print}} (\textcolor[HTML]{50a14f}{"Good bye!"})
\end{Verbatim}
\end{Code}

\begin{Code}[alt]
\begin{verbatim}
Current Letter : P
Current Letter : y
Current Letter : t
Current Letter : o
Current Letter : n
Current variable value : 9
Current variable value : 8
Current variable value : 7
Current variable value : 6
Current variable value : 4
Current variable value : 3
Current variable value : 2
Current variable value : 1
Current variable value : 0
Good bye!
\end{verbatim}
\end{Code}

Pass:
\begin{Code}
\begin{Verbatim}[]
\color{EFD}\textcolor[HTML]{9ca0a4}{\#!/usr/bin/python3}

\textcolor[HTML]{e45649}{for} \textcolor[HTML]{6a1868}{letter} \textcolor[HTML]{e45649}{in} \textcolor[HTML]{50a14f}{'Python'}:
   \textcolor[HTML]{e45649}{if} letter \textcolor[HTML]{e45649}{==} \textcolor[HTML]{50a14f}{'h'}:
      \textcolor[HTML]{e45649}{pass}
      \textcolor[HTML]{a626a4}{\textbf{print}} (\textcolor[HTML]{50a14f}{'This is pass block'})
   \textcolor[HTML]{a626a4}{\textbf{print}} (\textcolor[HTML]{50a14f}{'Current Letter :'}, letter)

\textcolor[HTML]{a626a4}{\textbf{print}} (\textcolor[HTML]{50a14f}{"Good bye!"})
\end{Verbatim}
\end{Code}

\begin{verbatim}
Current Letter : P
Current Letter : y
Current Letter : t
This is pass block
Current Letter : h
Current Letter : o
Current Letter : n
Good bye!
\end{verbatim}
\section{Assignment}
\label{sec:org36c0360}
This week, your assignment is to create a program which
\begin{enumerate}
\item Asks the user for input on three variables (x, y, and z)
\item Multiplies the first two variables
\item Compares the product of step 2 with the third value (if statement)
\item Prints
\texttt{Variable x+y is greater than z}
if its greater, else it should print
\texttt{Variable z is greater than x+y}
if its less
\end{enumerate}
\end{document}
