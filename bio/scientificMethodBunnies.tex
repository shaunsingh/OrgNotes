% Created 2021-09-29 Wed 09:46
% Intended LaTeX compiler: pdflatex
\documentclass{scrartcl}
\usepackage[utf8]{inputenc}
\usepackage[T1]{fontenc}
\usepackage{fontspec}
\usepackage{graphicx}
\usepackage{grffile}
\usepackage{longtable}
\usepackage{wrapfig}
\usepackage{rotating}
\usepackage[normalem]{ulem}
\usepackage{amsmath}
\usepackage{textcomp}
\usepackage{amssymb}
\usepackage{capt-of}
\usepackage[dvipsnames]{xcolor}
\usepackage[colorlinks=true, linkcolor=Blue, citecolor=BrickRed, urlcolor=PineGreen]{hyperref}
\usepackage{indentfirst}
\setmainfont[Ligatures=TeX]{Alegreya}
\setmonofont[Ligatures=TeX]{Liga SFMono Nerd Font}
% features: (acronym par-sep)
\newcommand{\acr}[1]{\protect\textls*[110]{\scshape #1}}
\newcommand{\acrs}{\protect\scalebox{.91}[.84]\hspace{0.15ex}s}
\setlength{\parskip}{\baselineskip}
\setlength{\parindent}{0pt}

% end features

%% make document follow Emacs theme

\definecolor{obg}{HTML}{fafafa}
\definecolor{ofg}{HTML}{383a42}

\pagecolor{obg}
\color{ofg}

% list labels

\definecolor{itemlabel}{HTML}{4078f2}

\renewcommand{\labelitemi}{\textcolor{itemlabel}{\textbullet}}
\renewcommand{\labelitemii}{\textcolor{itemlabel}{\normalfont\bfseries \textendash}}
\renewcommand{\labelitemiii}{\textcolor{itemlabel}{\textasteriskcentered}}
\renewcommand{\labelitemiv}{\textcolor{itemlabel}{\textperiodcentered}}

\renewcommand{\labelenumi}{\textcolor{itemlabel}{\theenumi.}}
\renewcommand{\labelenumii}{\textcolor{itemlabel}{(\theenumii)}}
\renewcommand{\labelenumiii}{\textcolor{itemlabel}{\theenumiii.}}
\renewcommand{\labelenumiv}{\textcolor{itemlabel}{\theenumiv.}}

% structural elements

\definecolor{documentTitle}{HTML}{a626a4}
\definecolor{documentInfo}{HTML}{a626a4}
\definecolor{level1}{HTML}{e45649}
\definecolor{level2}{HTML}{da8548}
\definecolor{level3}{HTML}{b751b6}
\definecolor{level4}{HTML}{6f99f5}
\definecolor{level5}{HTML}{bc5cba}
\definecolor{level6}{HTML}{9fbbf8}
\definecolor{level7}{HTML}{d292d1}
\definecolor{level8}{HTML}{d8e4fc}

\addtokomafont{title}{\color{documentTitle}}
\addtokomafont{author}{\color{documentInfo}}
\addtokomafont{date}{\color{documentInfo}}
\addtokomafont{section}{\color{level1}}
\newkomafont{sectionprefix}{\color{level1}}
\addtokomafont{subsection}{\color{level2}}
\newkomafont{subsectionprefix}{\color{level2}}
\addtokomafont{subsubsection}{\color{level3}}
\newkomafont{subsubsectionprefix}{\color{level3}}
\addtokomafont{paragraph}{\color{level4}}
\newkomafont{paragraphprefix}{\color{level4}}
\addtokomafont{subparagraph}{\color{level5}}
\newkomafont{subparagraphprefix}{\color{level5}}

% textual elements

\definecolor{link}{HTML}{4078f2}
\definecolor{cite}{HTML}{4aa8b0}
\definecolor{itemlabel}{HTML}{4078f2}
\definecolor{code}{HTML}{da8548}
\definecolor{verbatim}{HTML}{50a14f}

\renewcommand{\labelitemi}{\textcolor{itemlabel}{\textbullet}}
\renewcommand{\labelitemii}{\textcolor{itemlabel}{\normalfont\bfseries \textendash}}
\renewcommand{\labelitemiii}{\textcolor{itemlabel}{\textasteriskcentered}}
\renewcommand{\labelitemiv}{\textcolor{itemlabel}{\textperiodcentered}}

\renewcommand{\labelenumi}{\textcolor{itemlabel}{\theenumi.}}
\renewcommand{\labelenumii}{\textcolor{itemlabel}{(\theenumii)}}
\renewcommand{\labelenumiii}{\textcolor{itemlabel}{\theenumiii.}}
\renewcommand{\labelenumiv}{\textcolor{itemlabel}{\theenumiv.}}

\DeclareTextFontCommand{\texttt}{\color{code}\ttfamily}
\makeatletter
\def\verbatim@font{\color{verbatim}\normalfont\ttfamily}
\makeatother

% code blocks

\definecolor{codebackground}{HTML}{f6f6f6}
\colorlet{EFD}{ofg}
\definecolor{codeborder}{HTML}{f0f0f0}

%% end customisations

\author{Shaurya Singh}
\date{\today}
\title{Scientific Method HW \#1\\\medskip
\large Biology I}
\colorlet{greenyblue}{blue!70!green}
\colorlet{blueygreen}{blue!40!green}
\providecolor{link}{named}{greenyblue}
\providecolor{cite}{named}{blueygreen}
\hypersetup{
  pdfauthor={Shaurya Singh},
  pdftitle={Scientific Method HW \#1},
  pdfkeywords={},
  pdfsubject={},
  pdfcreator={Emacs 28.0.50 (Org mode 9.5)},
  pdflang={English},
  breaklinks=true,
  colorlinks=true,
  linkcolor=,
  urlcolor=link,
  citecolor=cite
}
\urlstyle{same}
\begin{document}

\maketitle
\setcounter{tocdepth}{2}
\tableofcontents


\section{Using Science Skills}
\label{sec:org6c3227d}
\begin{enumerate}
\item \textbf{Interpreting Graphics:} In Figure 1-1, which rabbit is the control?
\begin{itemize}
\item Rabit 2 is the control, since it doesn't have an ice pack
\end{itemize}
\item \textbf{lnterpreting Graphics:} In Figure 1-1, what is the variable in this
experiment?
\begin{itemize}
\item The independant variable is the temperature of the shaved patch on the
rabbit, and the dependant variable would be the color of fur produced.
\end{itemize}
\item \textbf{Formulating Hypotheses:} Before completing the experiment in Figure 1-1, the
scientist made a hypothesis. What is the hypothesis she is testing?
\begin{itemize}
\item She is testing the hypothesis ``If the patch of the rabbit is colder, then
the produced fur will be darker''
\end{itemize}
\item \textbf{Applying Concepts:} Why is Rabbit B essential to this experiment?
\begin{itemize}
\item Its important to have a control so we the researcher accurately test the
value of an independent variable on a dependent variable
\end{itemize}
\item \textbf{Drawing Conclusions:} Based on your observations of Figure 1-1, conclude what
effect temperature has on Himalayan rabbits.
\begin{itemize}
\item The Figure confirms our hypothesis, the colder rabbit produced darker skin,
so the enviornment does affect the color of rabbit fuduced darker skin, so
the enviornment does affect the color of rabbit fur
\end{itemize}
\end{enumerate}

\section{Bacteria Growth and Temperature}
\label{sec:orgd67b2f1}
\begin{enumerate}
\item \textbf{Classifying:} What variable did the researcher change during this experiment?
\begin{itemize}
\item The only variable ``changed'' was the time.
\end{itemize}
\item \textbf{Inferring:} What do the shapes of the curves tell you about the changes in
population size?
\begin{itemize}
\item As time increases, the rate of growth decreases
\end{itemize}
\item \textbf{Calculating:} For the bacteria kept at 15°C, how did population size change
during the experiment?
\begin{itemize}
\item The population size increased from around 3750 bacteria/ml of broth to
10000 bacteria/ml of broth
\end{itemize}
\item \textbf{Drawing Conclusions:} What effect did the different temperatures have on the
growth of the bacterial populations?
\begin{itemize}
\item A higher temperature results in a faster rate of growth
\end{itemize}
\item \textbf{Going Further:} Suppose some bacteria used in this experiment were kept at a
temperature of 700°C (the temperature of boiling water). Would you expect the
population sizes to increase even faster than at 15°C? Explain your
reasoning.
\begin{itemize}
\item The bacteria will likely \textbf{NOT} increase faster, as at temperatures over
65\textdegree{} C bacteria is rapidly killed.
\end{itemize}
\end{enumerate}
\end{document}
