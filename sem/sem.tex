% Created 2021-08-13 Fri 22:46
% Intended LaTeX compiler: pdflatex
\documentclass{scrartcl}
\usepackage[utf8]{inputenc}
\usepackage[T1]{fontenc}
\usepackage{fontspec}
\usepackage{graphicx}
\usepackage{grffile}
\usepackage{longtable}
\usepackage{wrapfig}
\usepackage{rotating}
\usepackage[normalem]{ulem}
\usepackage{amsmath}
\usepackage{textcomp}
\usepackage{amssymb}
\usepackage{dvipng}
\usepackage{chemfig}
\usepackage{mhchem}
\usepackage{capt-of}
\usepackage[dvipsnames]{xcolor}
\usepackage[colorlinks=true, linkcolor=Blue, citecolor=BrickRed, urlcolor=PineGreen]{hyperref}
\usepackage{indentfirst}
\setmainfont[Ligatures=TeX]{IBM Plex Sans}
\setmonofont[Ligatures=TeX]{Iosevka Nerd Font Mono}
% features: (acronym underline par-sep table)
\newcommand{\acr}[1]{\protect\textls*[110]{\scshape #1}}
\newcommand{\acrs}{\protect\scalebox{.91}[.84]{\hspace{0.15ex}s}}
\usepackage[normalem]{ulem}
\setlength{\parskip}{\baselineskip}
\setlength{\parindent}{0pt}

\usepackage{longtable}
\usepackage{booktabs}
% end features
\author{Shaurya Singh}
\date{\today}
\title{Ap Sem Summer Assignment}
\colorlet{greenyblue}{blue!70!green}
\colorlet{blueygreen}{blue!40!green}
\providecolor{link}{named}{greenyblue}
\providecolor{cite}{named}{blueygreen}
\hypersetup{
  pdfauthor={Shaurya Singh},
  pdftitle={Ap Sem Summer Assignment},
  pdfkeywords={},
  pdfsubject={},
  pdfcreator={Emacs 28.0.50 (Org mode 9.5)},
  pdflang={English},
  breaklinks=true,
  colorlinks=true,
  linkcolor=,
  urlcolor=link,
  citecolor=cite
}
\urlstyle{same}
\begin{document}

\maketitle
\setcounter{tocdepth}{2}
\tableofcontents


\section{Lens, Issues, and Articles}
\label{sec:org22e5037}
\begin{itemize}
\item Theme: Technology
\end{itemize}
\begin{longtable}{|p{2.5cm}|p{6.5cm}|p{1cm}|}
\toprule
Lens & Issue/Question & Article\\
\midrule
\endfirsthead
\multicolumn{3}{l}{Continued from previous page} \\
\toprule

Lens & Issue/Question & Article \\

\midrule
\endhead
\midrule\multicolumn{3}{r}{Continued on next page} \\
\endfoot
\endlastfoot
environmental & How can we use technology to efficiently generate and store energy? & \url{https://energystorage.org/why-energy-storage/technologies/}\\
\midrule
Scientific & How can we use technology to filter out artificial research papers? & \url{https://www.nature.com/articles/d41586-021-00733-5}\\
\midrule
Economic & How can we use AI to predict demand and reduce cost of goods? & \url{https://hbr.org/2016/11/the-simple-economics-of-machine-intelligence}\\
\midrule
Political and Historical & How can we use AI to analyze and predict the results of elections? & \url{https://www.wsj.com/articles/artificial-intelligence-shows-potential-to-gauge-voter-sentiment-11604704009}\\
\midrule
Artistic and Philosophical & How will artificially-generated art affect art prices in the future & \url{https://www.forbes.com/sites/jessedamiani/2020/09/21/in-this-exhibition-an-ai-dreams-up-imaginary-artworks-that-artist-alexander-reben-then-creates-irl/?sh=6c0d29e732e6}\\
\midrule
Cultural and Social & What are the effects of technology on a kids social life? & \url{https://www.linkedin.com/pulse/impacts-technology-culture-tradition-social-values-ashes-niroula}\\
\midrule
Futuristic & How will technology affect transportation in the future? & \url{https://capgemini-engineering.com/us/en/insight/how-technologies-will-change-the-future-of-transport/}\\
\midrule
Ethical & How can we prevent unethical robotics in the future? & \url{https://www.google.com/url?sa=t\&rct=j\&q=\&esrc=s\&source=web\&cd=\&cad=rja\&uact=8\&ved=2ahUKEwiq9c2RgKLyAhWTRjABHVk1BYMQFnoECAgQAQ\&url=https\%3A\%2F\%2Fwww.frontiersin.org\%2Farticles\%2F10.3389\%2Ffrobt.2017.00075\%2Ffull\&usg=AOvVaw2guSvyvgU4OWgLh\_aLAxnQ}\\
\bottomrule
\end{longtable}

\section{Choose one of the lenses with the best question which you have created. Read a few articles on the topic, and for two articles do the following:}
\label{sec:org644f31a}
\subsection{Article 1: [The fight against fake-paper factories that churn out sham science][\url{https://www.nature.com/articles/d41586-021-00733-5}]}
\label{sec:org857fe92}
\begin{enumerate}
\item Brief summary of the article
\begin{itemize}
\item Since last January, journals have retracted at least 370 papers that have
been publicly linked to paper mills. Many more retractions are expected to
follow; 15 are still under investigation. Research-integrity sleuths have
warned that some scientists buy papers from third-party firms to help their
careers. Image detectives who work under pseudonyms posted a list of more
than 400 published papers they said probably came from a paper mill in
January 2020. By March 2021, they had collectively listed more than 1,300
articles as possibly coming from paper mills. Around 26\% of the articles
that the sleuths alleged came from paper Mills have so far been retracted
or labelled with expressions of concern.
\end{itemize}

\item Authors Thesis:
\begin{itemize}
\item research-integrity sleuths have repeatedly warned that some scientists buy
papers from third-party firms to help their careers
\end{itemize}

\item To what extent are the authors claims valid?
\item What is one weakness with the author's claim?
\item Copy and paste 2 direct quotes that best represent this article .
\item Evaluate the article's effectiveness. Is it convincing?
\item Do you agree with the author?
\item Create an MLA works cited entry for each article
\end{enumerate}

\subsection{Article 2:  \url{https://www.christies.com/features/A-collaboration-between-two-artists-one-human-one-a-machine-9332-1.aspx}}
\label{sec:orgd80bbaf}
\begin{enumerate}
\item Brief summary of the article
\begin{itemize}
\item Christie's becomes the first auction house to offer a work of art created by an algorithm. Portrait of Edmond Belamy was created by Paris-based
collective Obvious. It is one of a group of portraits of the fictional
Belamy family created by Hugo Caselles-Dupré. The AI-generated artwork
sells for \$432,500, nearly 45 times its high estimate.
\end{itemize}

\item Authors Thesis:
\begin{itemize}
\item This portrait, however, is not the product of a human mind. It was created by an artificial intelligence, an algorithm defined by that algebraic formula with its many parentheses
\end{itemize}
\end{enumerate}
\begin{enumerate}
\item To what extent are the authors claims valid?
\item What is one weakness with the author's claim?
\item Copy and paste 2 direct quotes that best represent this article
\item Evaluate the article's effectiveness. Is it convincing?
\item Do you agree with the author?
\item Create an MLA works cited entry for each article
\end{enumerate}
\end{document}
